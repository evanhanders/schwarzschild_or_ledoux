\section{Theory \& Experiment}
\label{sec:theory}
The Schwarzschild criterion for convective stability is
\begin{equation}
    \yS \equiv \gradrad - \gradad < 0,
    \label{eqn:yS}
\end{equation}
whereas the Ledoux criterion for convective stability is
\begin{equation}
    \yL \equiv \yS +  \frac{\chi_\mu}{\chi_T}\gradmu < 0.
    \label{eqn:yL}
\end{equation}
The temperature gradient $\justgrad \equiv d \ln P / d \ln T$ (pressure $P$ and temperature $T$) is $\gradad$ for an adiabatic stratification and $\gradrad$ if all the flux is carried radiatively.
The Ledoux criterion includes the effects of the composition gradient $\gradmu = d\ln\mu/d\ln P$ (mean molecular weight $\mu$), where $\chi_T = (d\ln P / d\ln T)_{\rho,\mu}$ and $\chi_\mu = (d\ln P / d\ln\mu)_{\rho,T}$ (density $\rho$).

Stellar structure software instruments assume that convective boundaries coincide with sign changes of $\yL$ or $\yS$ \citep[][sec.~2]{mesa4}.
The various stability regimes that can occur in stars are described in section 3 and figure 3 of \citet{salaris_cassisi_2017}, but we note four important regimes here:
\begin{enumerate}
    \item Convection Zones (CZs): Regions with ${\yL > 0}$ are convectively unstable.
    \item Radiative Zones (RZs): Regions with $\yL \leq \yS < 0$ are always stable to convection.
        Other combinations of $\yL$ and $\yS$ \emph{may} also be stable RZs, as detailed below in \#3 and \#4.
    \item ``Semiconvection'' Zones (SZs): Regions with ${\yS > 0}$ but $\yL < 0$ are stablized by a composition gradient despite an unstable thermal stratification.
        These regions can be stable RZs or linearly unstable to oscillatory double-diffusive convection \citep[ODDC, see][chapters 2 and 4]{garaud_2018}.
    \item ``Thermohaline'' Zones: A strongly stable thermal stratification can overcome an unstable composition gradient in regions with $\yS < \yL < 0$.
        These regions can be stable RZs or linearly unstable to thermohaline mixing \citep[see][chapters 2 and 3]{garaud_2018}.
\end{enumerate}
In this letter, we study a three-layer 3D simulation of convection.
The initial structure of the simulation is an unstable CZ (bottom, \#1), a compositionally-stabilized SZ (middle, \#3), and a thermally stable RZ (top, \#2).
We examine how the boundary of the CZ evolves through entrainment.
In particular, we are interested in determining whether the heights at which $\yS = 0$ and $\yL = 0$ coincide on timescales that are long compared to the convective overturn timescale.

Our simulation uses the Boussinesq approximation, which is formally valid when motions occur on length scales much smaller than the pressure scale height.
This approximation fully captures nonlinear advective mixing near the CZ-SZ boundary, which is our primary focus.
Our simulations use a height-dependent $\gradrad$ and buoyancy is determined by a combination of the composition and the temperature stratification, so $\yS$ and $\yL$ are determined independently and self-consistently.
Our simulation length scales are formally much smaller than a scale height, but a useful heuristic is to think of our 3D convection zone depth (initially 1/3 of the simulation domain) as being analogous to the mixing length in a 1D stellar evolution model.
For details on our model setup and Dedalus \citep{burns_etal_2020} simulations, we refer the reader to appendices \ref{app:model} and \ref{app:simulation_details}.
