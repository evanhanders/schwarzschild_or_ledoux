\section{Theory}
\label{sec:theory}

\begin{itemize}
\item The stability of a convective region can instantaneously be determined using the Schwarzschild criterion,
\begin{equation}
    y_S = \gradrad - \gradad,
\end{equation}
or the Ledoux criterion,
\begin{equation}
y_L = \gradrad - \gradL,
\end{equation}
with the unstable convection zone contained within the region where $y < 0$ \citep[see e.g., section 2 of][]{mesa4}.
\item The \emph{instantaneous} stability changes over the course of many convective freefall times, and stellar evolution timesteps generally span many convective overturn times.
\item The goal of this work is to find if $y_S$ and $y_L$ evolve toward the same value, and whether they evolve toward the initial boundary as determined by $y_S$ or $y_L$.
\item We will study a numerical simulation with three layers: a convective layer $y_s < 0$ and $y_L < 0$, a ``semiconvective'' \citep[see section 4 of][]{mesa2} \citep{salaris_cassisi_2017} layer with $y_S < 0$ but $y_L > 0$ \citep[which is stable to ODDC][]{garaud_2018}, and a fully stable layer with $y_L > 0$ and $y_S > 0$.
\item While ``semiconvective'' layers are often unstable to doubly-diffusive instabilities in stellar regimes \citep{moore_garaud_2016}, our purpose in this work is to demonstrate that regardless of whether these layers are doubly-diffusive unstable, they are entrained and destroyed by neighboring convective zones.
\end{itemize}

