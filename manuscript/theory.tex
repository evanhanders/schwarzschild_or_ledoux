\section{Theory}
\label{sec:theory}
The stability of a convective region can instantaneously be determined using the Schwarzschild criterion,
\begin{equation}
    \yS = \gradrad - \gradad,
\end{equation}
or the Ledoux criterion,
\begin{equation}
    \yL = \gradrad - \gradL.
\end{equation}
Here, $\mathcal{Y}$ is the discriminant \citep[e.g.,][sec.~2]{mesa4} or superadiabaticity.
In stellar structure codes, it is assumed that regions with $\mathcal{Y} < 0$ are convection zones (CZs) while reagions with $\mathcal{Y} > 0$ are radiative zones (RZs).

When both stability criteria are considered, more interesting behavior can occur in certain physical regimes.
A region is a CZ when both $\yS < 0$ and $\yL < 0$, and a region is an RZ when both $\yS > 0$ and $\yL \geq \yS$.
However, a ``semiconvective'' region is one that is Schwarzschild unstable ($\yS < 0$) and Ledoux stable ($\yL > 0$), and these regions can either be linearly stable or can exhibit overstable doubly-diffusive convection \citep[ODDC, see][chapter 2]{garaud_2018}. CITES
In this paper, we construct our initial conditions in our simulations so that the semiconvective regions are stable to ODDC, so that we can study the growth of convection zones through entrainment.
Some RZs with $\yS > \yL$ (stable regions with unstable composition gradients) are also unstable to thermohaline or fingering convection \citep[see][chapter 3]{garaud_2018} CITE, but we do not study these regimes in this work.

In this work, we solve the simplest possible convective system.
We utilize the Boussinesq approximation, which assumes that the depth of the layer being studied is much smaller than the local scale height.
Since we are studying thin regions near convective boundaries, this assumption is OK.
The relevant physics for this problem are included ($\gradrad$ varies with height, buoyancy is determined both by the composition $C$ and the temperature stratification $T$), so $\yS$ and $\yL$ are meaningfuly defined and distinct from one another when composition gradients are present.
For details on our model setup and Dedalus simulations, we refer the reader to appendices \ref{app:model} and \ref{app:simulation_details}.


\begin{enumerate}
    \item Define point of neutral buoyancy (need to come up with good notation for this stuff or get it from literature), mention that we use that to define convective regions in this work because of (a) 'noise' in $N^2$ in bulk-CZ
    \item I just realized that if I define y as in stellar structure (using gradrad instead of grad), then I wouldn't have to worry about noise in the mid-CZ because gradrad has a clean profile. And I think using y < 0 as in stellar structure would work straightforwardly. I should try that out. I could drop #2 above if I do that.
\end{enumerate}

\begin{itemize}
\item The \emph{instantaneous} stability changes over the course of many convective freefall times, and stellar evolution timesteps generally span many convective overturn times.
\item The goal of this work is to find if $y_S$ and $y_L$ evolve toward the same value, and whether they evolve toward the initial boundary as determined by $y_S$ or $y_L$.
\item We will study a numerical simulation with three layers: a convective layer $y_s < 0$ and $y_L < 0$, a ``semiconvective'' \citep[see section 4 of][]{mesa2} \citep{salaris_cassisi_2017} layer with $y_S < 0$ but $y_L > 0$ \citep[which is stable to ODDC][]{garaud_2018}, and a fully stable layer with $y_L > 0$ and $y_S > 0$.
\item While ``semiconvective'' layers are often unstable to doubly-diffusive instabilities in stellar regimes \citep{moore_garaud_2016}, our purpose in this work is to demonstrate that regardless of whether these layers are doubly-diffusive unstable, they are entrained and destroyed by neighboring convective zones.
\end{itemize}

