
\section{Introduction}
\label{sec:introduction}
Observations tell us that we do not understand the positioning of convective boundaries in stars.
For example, models and observations disagree about the sizes of convective cores \citep{claret_torres_2018, viani_basu_2020, pedersen_etal_2021, johnston_2021}, lithium abundances in solar-type stars \citep{pinsonneault_1997, sestito_randich_2005, carlos_etal_2019, dumont_etal_2021}, and the sound speed at the base of the Sun's convection zone \citep[see][Sec.~7.2.1]{basu_2016}.
Improperly estimating convective boundary locations can have important impacts across astrophysics such as by affecting the mass of stellar remnants \citep{farmer_etal_2019, mehta_etal_2022} and the inferred radii of exoplanets \citep{basu_etal_2012, morrell_2020}.

While convective boundary mixing (CBM) has many uncertainties, the most fundamental question is: what determines the location of convection zone boundaries? 
Some authors evaluate the \emph{Schwarzschild criterion}, which determines where the temperature and pressure stratification within a star are stable or unstable.
Others evaluate the \emph{Ledoux criterion}, which accounts for stability or instability due to compositional stratification \citep[e.g., the variation of helium abundance with pressure; see][chapter 3, which reviews these criteria]{salaris_cassisi_2017}.
Recent authors state that these criteria should be equivalent at a convective boundary according to mixing length theory \citep{gabriel_etal_2014, mesa4, mesa5}, but in practice these criteria are often different at convective boundaries in stellar evolution software instruments, and a variety of workarounds have been proposed to address this~\citep{mesa4,mesa5}.

There is still disagreement regarding which stability criterion to employ~\citep[discussed in][chapter 2]{kaiser_etal_2020}.
Multi-dimensional simulations show that convection zones with Ledoux-stable boundaries expand by entraining compositionally-stable regions \citep{meakin_arnett_2007, woodward_etal_2015, jones_etal_2017, cristini_etal_2019, fuentes_cumming_2020, andrassy_etal_2020, andrassy_etal_2021}.
It is unclear from past 3D simulations whether that entrainment should stop, leading to uncertainty in how to include entrainment in 1D models \citep{staritsin_2013, scott_etal_2021}.

In this letter, we present a simple 3D hydrodynamical simulation that demonstrates that convection zones with Ledoux-stable but Schwarzschild-unstable boundaries entrain material until the Ledoux and Schwarzschild criteria agree on the location of the convective boundary.
Therefore, in 1D stellar evolution models, when evolutionary timescales are much larger than the convective overturn timescale \citep[such as on the main sequence, see][]{georgy_etal_2021}, the Schwarzschild criterion describes the location of the convective boundary, and Ledoux and Schwarzschild should agree if properly implemented.
We discuss these criteria in Sec.~\ref{sec:theory}, display simulations in Sec.~\ref{sec:results}, and briefly discuss in Sec.~\ref{sec:conclusions}.
