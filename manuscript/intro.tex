
\section{Introduction}
\label{sec:introduction}
We do not understand convective boundaries in stars.
There are discrepancies between models and observations regarding the sizes of convective cores \citep{claret_torres_2018, viani_basu_2020, pedersen_etal_2021, johnston_2021}, lithium abundances in solar-type stars \citep{pinsonneault_1997, sestito_randich_2005, carlos_etal_2019, dumont_etal_2021}, and the sound speed at the base of the Sun's convection zone \citep[see][Sec.~7.2.1]{basu_2016}.
Incorrect convective boundary locations can have important impacts across astrophysics such as affecting the mass of stellar remnants \citep{farmer_etal_2019, mehta_etal_2022} and the inferred radii of exoplanets \citep{basu_etal_2012, morrell_2020}.

While convective boundary mixing (CBM) has many uncertainties, the most fundamental question is: what determines the location of convection zone boundaries? 
Some stellar evolution models determine the location of the convection zone boundary using the \emph{Schwarzschild criterion}, by comparing the radiative and adiabatic temperature gradients.
In other models, the convection zone boundary is determined by using the \emph{Ledoux criterion}, which also accounts for compositional stratification \citep[][chapter 3, reviews these criteria]{salaris_cassisi_2017}.
Recent work states that these criteria should be equivalent at a convective boundary according to mixing length theory \citep{gabriel_etal_2014, mesa4, mesa5}, but in practice these criteria are often different at convective boundaries in stellar evolution software instruments, and this has led to a variety of algorithms for determining boundary locations \citep{mesa4,mesa5}.

As there is still disagreement regarding which stability criterion is appropriate for 1D modeling~\citep[see][chapter 2]{kaiser_etal_2020}, insight can be gained from studying multi-dimensional simulations.
Such simulations show that a convection zone adjacent to a Ledoux-stable region can expand by entraining material from the stable region \citep{meakin_arnett_2007, woodward_etal_2015, jones_etal_2017, cristini_etal_2019, fuentes_cumming_2020, andrassy_etal_2020, andrassy_etal_2021}.
However, past simulations have not achieved a statistically-stationary state, leading to uncertainty in how to include entrainment in 1D models \citep{staritsin_2013, scott_etal_2021}.

In this letter, we present a 3D hydrodynamical simulation that demonstrates that convection zones adjacent to regions that are Ledoux-stable but Schwarzschild-unstable will entrain material until the adjacent region is stable by both criteria.
Therefore, in 1D stellar evolution models, the Schwarzschild criterion correctly determines the location of the convective boundary when evolutionary timescales are much larger than the convective overturn timescale \citep[e.g., on the main sequence;][]{georgy_etal_2021}.
When correctly implemented, the Ledoux criterion should return the same result \citep{gabriel_etal_2014}.
We discuss these criteria in Sec.~\ref{sec:theory}, describe our simulation in Sec.~\ref{sec:results}, and briefly discuss the implications of our results for 1D stellar evolution models in Sec.~\ref{sec:conclusions}.
