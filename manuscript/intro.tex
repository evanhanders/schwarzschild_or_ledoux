
\section{Introduction}
\label{sec:introduction}

\begin{itemize}
\item Observations tell us that we don't understand the mixing near convective boundaries.
\item One fundamental question is: what stability criterion should be used? Define Schwarzschild and Ledoux boundaries.
\item Naive implementation e.g., in \citet{mesa2} run into problems in the context of MLT logic.
\item These naive implementations also are at odds with simulations: high entrainment rates (citations), and entrainment theories suggest that ledoux-stable but schwarzschild-unstable regions should be fully mixed \citep{fuentes_cumming_2020}, more citations.
\item Furthermore, \citet{gabriel_etal_2014} point out that a naive use of the Ledoux criterion is logically inconsistent within the framework of mixing length theory.
\item New algorithms \citep{mesa4, mesa5} have been developed which handle boundaries self-consistently and essentially make Schwarzschild and Ledoux criterion the same.
\item This theoretical discussion has occurred in the context of theoretical arguments and 1D models, but convection is a three-dimensional process.
\item Fundamental, simple experiments which demonstrate the efficacy of these new algorithms, and whether convection prefers the Schwarzschild or Ledoux stability criterion, are lacking.
\item In this work, we present a simple three-dimensional experiment that demonstrates that regions that are stable by the Ledoux criterion are fragile, and that, at late times, the Ledoux and Schwarzschild criterion both find the same edge of the convection zone.
\end{itemize}

