\section{Conclusions \& Discussion}
\label{sec:conclusions}

In this letter, we presented a 3D simulation of a convection zone and its boundary.
The convective boundary was initially compositionally stable but weakly thermally unstable (Ledoux stable but Schwarzschild unstable).
Entrainment caused the convective boundary to advance until the boundary was stable by both the Schwarzschild and Ledoux criteria.

This simulation demonstrates that the Ledoux criterion \emph{instantaneously} describes the location of a convective boundary.
However, when the convective overturn timescale is short compared to the evolutionary timescale $t_{\rm{evol}} \gg t_{\rm{conv}}$, the statistically-stationary location of the convective boundary will coincide with the Schwarzschild boundary.
These 3D dynamics support the claim that ``logically consistent'' implementations of mixing length theory \citep{gabriel_etal_2014, mesa4, mesa5} should have convective boundaries which are Schwarzschild-stable.
Modern algorithms like the MESA software instrument's ``convective pre-mixing'' \citep[CPM,][]{mesa5} should agree with our results.
The results of 1D stellar evolution calculations should not depend on the choice of stability criterion used when $t_{\rm{evol}} \gg t_{\rm{conv}}$.

We note briefly that many SZs (the middle layer of our simulations) in stars are unstable to overstable doubly-diffusive convection (ODDC).
ODDC mixes composition gradients even more rapidly than the entrainment studied here, and has been studied extensively in local simulations \citep{mirouh_etal_2012, wood_etal_2013, xie_etal_2017}; see the review of \citet{garaud_2018}.
\citet{moore_garaud_2016} apply ODDC to the regions outside core convection zones in main sequence stars, and their results suggest that ODDC formulations should be widely included in stellar models.

For stages in stellar evolution where $t_{\rm{conv}} \sim t_{\rm{evol}}$, implementations of time-dependent convection \citep[TDC,][]{tdc_1986} should be employed to properly capture convective dynamics and the advancement of convective boundaries.
The advancement of convective boundaries by e.g., entrainment in TDC implementations should be informed by time-dependent theories and simulations \citep[e.g.,][]{turner_1968, fuentes_cumming_2020}.

The purpose of this study was to understand how the Ledoux boundary location evolves over time, and whether it coincides with the Schwarzschild boundary at late times.
A detailed examination of convective overshoot is beyond the scope of this work \citep[but see e.g.,][]{korre_etal_2019}.
We furthermore constructed the simulations in this work to have negligible convective penetration per \citet{anders_etal_2022}.
Finally, in stars, $\gradrad$ and the Schwarzschild boundary location depend upon $\mu$, but we made the assumption that the radiative conductivity and $\gradrad$ do not depend on $\mu$ for simplicity.
The nonlinear feedback between these effects should be studied in future work, but we do not expect that the fundamental takeaways of this work should change.

In summary, we find that the Schwarzschild criterion provides the location of the convective boundary in a statistically stationary state; in this final state, the Ledoux and Schwarzschild criteria are degenerate.
