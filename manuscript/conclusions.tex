\section{Conclusions}
\label{sec:conclusions}

\begin{itemize}
    \item We simulated X.
    \item These simulations demonstrate that the Schwarzschild criterion provides a good estimate of the boundary of the unstable convection zone.
    \item This analysis ignores secondary effects \citep[e.g., convective penetration][because we designed our experiments to minimize these effects]{anders_etal_2022}.
    \item These results provide evidence from 3D dynamical simulations that modern implementations of MLT \citep{mesa4, mesa5} which are more logically consistent \citep{gabriel_etal_2014} are more accurate.
    \item Our results demonstrate that while the Ledoux criterion is the proper \emph{instantaneous} criterion for the boundary of convective regions, entrainment will erode stable composition gradients at convective boundaries.
        For stages in stellar evolution where $t_{\rm{conv}} \ll t_{\rm{evolution}}$, stellar evolution codes should use the Schwarzschild boundary; for stages where $t_{\rm{conv}} \lesssim t_{\rm{evolution}}$, future 3D numerical simulations should develop a more complete and parameterized theory of entrainment at convective boundaries.
\end{itemize}

