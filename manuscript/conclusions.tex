\section{Conclusions \& Discussion}
\label{sec:conclusions}

In this letter, we present a 3D simulation of a convection zone adjacent to a compositionally stable and weakly thermally unstable region.
This region is stable according to the Ledoux criterion, but unstable according to the Schwarzschild criterion.
Overshooting convective motions entrain the entire Schwarzschild-unstable region until the Schwarzschild and Ledoux criterion both predict the same boundary of the convection zone.

This simulation demonstrates that, while the Ledoux criterion \emph{instantaneously} predicts the location of the convective boundary, on evolutionary timescales ($t_{\rm{evol}} \gg \Ro^{-1}\mS t_{\rm{dyn}}$, see Sec.~\ref{sec:results}) the convective boundary is given by the Schwarzschild criterion.
Our 3D simulation supports the claim that ``logically consistent'' implementations of mixing length theory \citep{gabriel_etal_2014, mesa4, mesa5} should have convective boundaries which are Schwarzschild-stable.
E.g., the MESA software instrument's ``convective pre-mixing'' \citep[CPM,][]{mesa5} is consistent with our simulation.
The results of 1D stellar evolution calculations should not depend on the choice of stability criterion used when $t_{\rm{evol}} \gg \Ro^{-1}\mS t_{\rm{dyn}}$.

In many stars, the SZ is unstable to oscillatory double-diffusive convection (ODDC).
ODDC mixes composition gradients even more rapidly than the entrainment studied here, and has been studied extensively in local simulations \citep{mirouh_etal_2012, wood_etal_2013, xie_etal_2017}; see the review of \citet{garaud_2018}.
\citet{moore_garaud_2016} apply ODDC to the regions outside core convection zones in main sequence stars, and their results suggest that ODDC formulations should be widely included in stellar models.
This letter shows that ODDC-unstable zones should not form at convective boundaries.
However, if they did, they would likely be rapidly mixed by a combination of ODDC and entrainment.

For stages in stellar evolution where $\Ro^{-1}\mS t_{\rm{dyn}} \sim t_{\rm{evol}}$, implementations of time-dependent convection \citep[TDC,][]{tdc_1986} are required properly capture convective dynamics.
These evolutionary stages should also implement time-dependent entrainment models to properly advance convective boundaries \citep[e.g.,][]{turner_1968, fuentes_cumming_2020}.

\citet{anders_etal_2022} showed convective motions can extend significantly into the radiative zoens of stars via ``penetrative convection.''
In this work, we used parameters which do not have significant penetration.
This can be seen because while the composition is well-mixed above the convective boundary, the thermal structure does not mix to the adiabatic.

We assume that the radiative conductivity and $\gradrad$ do not depend on $\mu$ for simplicity.
The nonlinear feedback between these effects should be studied in future work, but we expect that our conclusions are robust.

%Possibly delete?
%In summary, we find that the Ledoux criterion provides the instantaneous location of the convective boundary, and the Schwarzschild criterion provides the location of the convective boundary in a statistically stationary state; in this final state, the Ledoux and Schwarzschild criteria agree.
