\section{Model \& Initial Conditions}
\label{app:model}
In this work we study the simplest possible system: incompressible, Boussinesq convection with a composition field and a height-varying background radiative conductivity, similar to that used in \citet{fuentes_cumming_2020, anders_etal_2022}.
These equations are
\begin{align}
    &\grad\dot\vec{u} = 0
        \label{eqn:incompressible_dimensional}, \\
    &\partial_t \vec{u}\dot\grad\vec{u} = -\frac{1}{\rho_0}\grad p + \frac{\rho_1}{\rho_0}\vec{g} + \nu\grad^2\vec{u}
        \label{eqn:momentum_dimensional}, \\
    &\partial_t T + \vec{u}\dot\grad T + w\gradad + \grad\dot[-\kappa_{T,0}\grad\bar{T}] = \kappa_T\grad^2 T'
        \label{eqn:temperature_dimensional}, \\
    &\partial_t C + \vec{u}\dot\grad C = \kappa_{C,0}\grad^2\bar{C} + \kappa_C\grad^2 C'
        \label{eqn:composition_dimensional}, \\
    &\frac{\rho_1}{\rho_0} = -|\alpha|T + |\beta|C
        \label{eqn:boussinesq}.
\end{align}
Here, $\vec{u}$ is the vector velocity, $T$ is the temperature, $C$ is the composition, $\rho_0$ is the constant background density, $p$ is the kinematic pressure which enforces Eqn.~\ref{eqn:incompressible_dimensional}, $\rho_1$ are density fluctuations which act only on the buoyant term, and $\alpha$ and $\beta$ are the thermal and compositional expansion coefficients, and $\gradad$ is the adiabatic gradient.
Diffusive terms are controlled by the kinematic viscosity $\nu$, as well as the thermal diffusivity $\kappa_T$ and compositional diffusivity $\kappa_C$.
On the horizontally-invariant ($n_x = 0$ and $n_y = 0$) mode, we use a height-depended thermal diffusion coefficient $\kappa_{T,0}$ (which allows $\gradrad$ to vary with height) and a lower compositional diffusivity $\kappa_{C,0} < \kappa_C$ to ensure that the evolution of the mean composition profile is due to advection rather than diffusion.

We nondimensionalize Eqns.~\ref{eqn:incompressible_dimensional}-\ref{eqn:boussinesq} according to
\begin{equation}
\begin{split}
    &T^* = (\Delta T)T,\qquad
    C^* = (\Delta C)C,\qquad
    \partial_{t^*} = \tau_{\rm{ff}}^{-1}\partial_t,\qquad\,\,\,
    \grad^* = L_s^{-1} \grad,\qquad
    p^* = \rho_0 u_{\rm{ff}}^2\varpi,
\\
    &\vec{u}^* = u_{\rm{ff}}\vec{u} = \frac{L_s}{\tau_{\rm{ff}}} \vec{u}, \qquad\qquad
    \tau_{\rm{ff}} = \left(\frac{L_s}{|\alpha| g \Delta T}\right)^{1/2}, \qquad\qquad
    \kappa_{T,0}^* = (L_s^2 \tau_{\rm{ff}}^{-1})$\kappa_{T,0}$.
%    \kappa_T^* = (L_s^2 \tau_{\rm{ff}}^{-1})$\kappa_T$,\qquad
%    \kappa_C^* = (L_s^2 \tau_{\rm{ff}}^{-1})$\kappa_C$.
\end{split}
\end{equation}
For convenience, here we define quantities with $*$ (e.g., $T^*$) as being the ``dimensionful'' quantities of Eqns.~\ref{eqn:incompressible_dimensional}-\ref{eqn:boussinesq}.
Henceforth, quantities without $*$ (e.g., $T$) are dimensionless.
Here, $L_s$ is the length scale of the initial Schwarzschild-unstable convection zone and $\tau_ff$ is the buoyant freefall timescale.
The temperature and composition are set by the destabilizing radiative temperature gradient $\Delta T = L_s (\partial_z T + \grad_{\rm{ad}})$ and the stabilizing composition gradient ($\Delta C = L_s \partial_z C$).
Within this nondimensionalization, the dynamical control parameters are
\begin{equation}
\begin{split}
    &\mP = \frac{u_{\rm{ff}} L_s}{\kappa_T},\qquad
    \Ro = \frac{|\alpha|\Delta T}{|\beta|\Delta C},\qquad
    \Pran = \frac{\nu}{\kappa_T},\qquad
    \tau = \frac{\kappa_C}{\kappa_T},\qquad
    \tau_0 = \frac{\kappa_{C,0}}{\kappa_T}
\end{split}
\end{equation}
The dimensionless equations of motion are
\label{sec:simulation_details}
\begin{align}
    &\grad\dot\vec{u} = 0 
        \label{eqn:incompressible} \\
    &\partial_t \vec{u} + \vec{u}\dot\grad\vec{u} = -\grad \varpi + (T - \Ro^{-1}C) \hat{z} + \frac{\Pran}{\mP}\grad^2 \vec{u}
        \label{eqn:momentum} \\
    &\partial_t T + \vec{u}\dot\grad T + w \grad_{\rm{ad}}  + \grad\dot[-\kappa_{T,0} \grad \overline{T}] = \frac{1}{\mP}\grad^2 T'.
        \label{eqn:temperature}, \\
    &\partial_t C + \vec{u}\dot\grad C = -\frac{\tau_0}{\mP}\grad^2\bar{C} + \frac{\tau}{\mP}\grad^2 C'.
        \label{eqn:composition}
\end{align}
We define the thermal and compositional gradients
\begin{equation}
    \gradT \equiv -\frac{\partial T}{\partial z},\qquad
    \gradC \equiv -\Ro^{-1} \frac{\partial C}{\partial z},
\end{equation}
and stability is determined by the sign of the {\brunt} frequency,
\begin{equation}
    N^2 = N_{\rm{structure}}^2 + N_{\rm{composition}}^2,\,\,\text{with}\,\,
    N_{\rm{structure}}^2 = -(\gradT - \gradad),\,\,
    N_{\rm{composition}}^2 = \gradC,
\end{equation}
where $N^2 > 0$ is buoyantly stable, so the stability criterion is $\gradC - (\gradT - \gradad) > 0$, as in stellar models \citep{salaris_cassisi_2017}.

In this work, we study a three-layer model in $z = [0, 3]$.
We want to construct a simulation with
\begin{equation}
    N^2 = 
    \begin{cases}
        -1           & z \leq 1 \\
        R_0^{-1} - 1 & 1 < z \leq 2 \\
        R_0^{-1}     & 2 < z
    \end{cases},\qquad
    N_{\rm{composition}}^2 = 
    \begin{cases}
        0        & z \leq 1 \\
        R_0^{-1} & 1 < z \leq 2 \\
        0        & 2 < z
    \end{cases}, \qquad
    N_{\rm{structure}}^2 = 
    \begin{cases}
        -1           & z \leq 1 \\
        -1           & 1 < z \leq 2 \\
        R_0^{-1}     & 2 < z
    \end{cases}
\label{eqn:initial_brunt}
\end{equation}
To achieve this, we set $\partial_z C = - \Ro N_{\rm{composition}}^2$ and $\partial_z T = N_{\rm{structure}}^2 - \gradad$, where we set $\gradad = 5[\Ro^{-1} - 2]$ as a constant so that $\gradad > 0$ for all values of $\Ro$ studied.
We furthermore enforce that $\gradT = \gradrad$ in the initial state, where
\begin{equation}
    \gradrad = -\frac{F_{\rm{tot}}}{\kappa_{T,0}},
\end{equation}
is the radiative gradient and $\Ftot$ is the total vertical energy flux through the system.
We set the total flux $\Ftot = -(1 + \gradad) / \mP$ and the convective flux $F_{\rm{conv}} = 1 / \mP$, so $\kappa_{T,0}(z) = -((1 + \gradad)/\partial_z T)\mP^{-1} $.
%and $\gradT = \gradrad$ if conduction carries all of the energy flux.
%We fix the flux carried by convection to be
%\begin{equation}
%    \Fconv = \kappa_{T,0},
%\end{equation}
%so the total flux through the system is $F_{\mathrm{tot}} = \Fconv(\gradad + 1)$.
%We set $\gradT = \gradrad$ in the initial state.
