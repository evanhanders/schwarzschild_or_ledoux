\section{Model \& Initial Conditions}
\label{app:model}
We study incompressible, Boussinesq convection \editone{in which we evolve both} temperature $T$ and concentration $\mu$.
The nondimensional equations of motion are
\begin{align}
    &\grad\dot\vec{u} = 0 
        \label{eqn:incompressible} \\
    &\partial_t \vec{u} + \vec{u}\dot\grad\vec{u} + \grad \varpi = \left(T - \frac{\mu}{\Ro}\right) \hat{z} + \frac{\Pran}{\rm{Pe}}\grad^2 \vec{u}
        \label{eqn:momentum}, \\
    &\partial_t T + \vec{u}\dot(\grad T - \hat{z}\,\partial_z T_{\rm{ad}})   = \grad\dot[\kappa_{T,0} \grad \overline{T}] +  \frac{1}{\rm{Pe}}\grad^2 T'
        \label{eqn:temperature},\\
    &\partial_t \mu + \vec{u}\dot\grad \mu = \frac{\tau_0}{\rm{Pe}}\grad^2\bar{\mu} + \frac{\tau}{\rm{Pe}}\grad^2 \mu',
        \label{eqn:composition}
\end{align}
where $\vec{u}$ is velocity.
Overbars denote horizontal averages and primes denote fluctuations around that average such that $T = \bar{T} + T'$.
The adiabatic temperature gradient is $\partial_z T_{\rm{ad}}$ and the nondimensional control parameters are
\begin{equation}
\begin{split}
    &\mathrm{Pe} = \frac{u_{\rm{ff}} h_{\rm{conv}}}{\kappa_T},\qquad
    \Ro = \frac{|\alpha|\Delta T}{|\beta|\Delta \mu},\qquad\\
    &\Pran = \frac{\nu}{\kappa_T},\qquad
    \tau = \frac{\kappa_\mu}{\kappa_T},\qquad
\end{split}
\end{equation}
where the nondimensional freefall velocity is $u_{\mathrm{ff}} = \sqrt{|\alpha|g h_{\rm{conv}}\Delta T}$ (with gravitational acceleration $g$), $h_{\rm{conv}}$ is the initial depth of the convection zone, \editone{the constant} $\Delta \mu$ is the \editone{initial} composition change across the Ledoux stable region, \editone{the constant} $\Delta T = h_{\rm{conv}}(\partial_z T_{\rm{rad}} - \partial_z T_{\rm{ad}})$ is the \editone{initial} superadiabatic temperature scale of the convection zone, \editone{$\alpha \equiv (\partial \ln \rho / \partial T)|_{\mu}$ and $\beta \equiv (\partial \ln \rho / \partial \mu)|_{T}$} are respectively the coefficients of expansion for $T$ and $\mu$, the viscosity is $\nu$, $\kappa_T$ is the thermal diffusivity, and $\kappa_\mu$ is the compositional diffusivity.
\editone{In stellar structure modeling, $\Ro = |N_{\rm{structure}}^2/N_{\rm{composition}}^2|$ is the ratio of respectively the thermal and compositional components of the \brunt$\,$frequency as measured in a semiconvection zone or thermohaline zone.}
Eqns.~\ref{eqn:incompressible}-\ref{eqn:composition} are identical to Eqns.~2-5 in \citet{garaud_2018}, except we modify the diffusion coefficients acting on $\bar{T}$ ($\kappa_{T,0}$) and $\bar{\mu}$ ($\tau_0$).
By doing this, we keep the turbulence (Pe) uniform throughout the domain while also allowing the radiative temperature gradient $\partial_z T_{\rm{rad}} = -\rm{Flux}/\kappa_{T,0}$ to vary with height.
We furthermore reduce diffusion on $\bar{\mu}$ to ensure its evolution is due to advection.

We define the Ledoux and Schwarzschild discriminants
\begin{equation}
    \yS = \left(\frac{\partial T}{\partial z}\right)_{\rm{rad}} - \left(\frac{\partial T}{\partial z}\right)_{\rm{ad}},\,\,
    \yL = \yS - \Ro^{-1} \frac{\partial \mu}{\partial z},
\end{equation}
and in this nondimensional system the square {\brunt} frequency is $N^2 = -\yL$.

We study a three-layer model with $z \in [0, 3]$,
\begin{align}
    &\left(\frac{\partial T}{\partial z}\right)_{\rm{rad}} = 
    \left(\frac{\partial T}{\partial z}\right)_{\rm{ad}} + 
    \begin{cases}
        -1           & z \leq 2 \\
        10\Ro^{-1}     & z > 2
    \end{cases},
    \label{eqn:initial_T}
    \\
    &\frac{\partial \mu_0}{\partial z} = 
    \begin{cases}
        0        & z \leq 1 \\
        -1       & 1 < z \leq 2 \\
        0        & 2 > z
    \end{cases},
    \label{eqn:initial_mu}
\end{align}
We set $(\partial T / \partial z)_{\rm{ad}} = -1 - 10\Ro^{-1}$.
The intial temperature profile has $\partial_zT_0 = \partial_z T_{\rm{rad}}$ everywhere except between $z = [0.1, 1]$ where $\partial_zT_0 = \partial_z T_{\rm{ad}}$.
\editone{Step functions are not well represented in pseudospectral codes, so we use smooth heaviside functions (Eqn.~\ref{eqn:heaviside}) to construct these piecewise functions.}
To obtain $T_0$, we numerically \editone{integrate the smooth} $\partial_z T_0$ profile with $T_0(z=3) = 1$.
To obtain $\mu_0$, we numerically integrate \editone{the smooth} Eqn.~\ref{eqn:initial_mu} with $\mu_0(z=0) = 0$.

For boundary conditions, we hold ${\partial_z T = \partial_z T_0}$ at $z = 0$, $T = T_0$ at $z = 3$, and we set ${\partial_z \mu = \hat{z}\dot\vec{u} = \hat{x}\dot\partial_z\vec{u} = \hat{y}\dot\partial_z\vec{u}(z=0) = \hat{y}\dot\partial_z\vec{u}(z=3) = 0}$ at $z = [0,3]$.
The simulation in this work uses $\rm{Pe} = 3.2 \times 10^3$, $\Ro^{-1} = 10$, $\rm{Pr} = \tau = 0.5$, $\tau_0 = 1.5 \times 10^{-3}$, and ${\kappa_{T,0} = \rm{Pe}^{-1}[(\partial T/\partial z)_{\rm{rad}}|_{z=0}] / (\partial T/\partial z)_{\rm{rad}}}$
\editone{
    The convective cores of main sequence stars with ${2M_\odot \lesssim M_* \lesssim 10M_\odot}$ have $\rm{Pe} = \mathcal{O}(10^6)$, $\tau \approx \rm{Pr} = \mathcal{O}(10^{-6})$, and stiffnesses of $\mS = \mathcal{O}(10^{6-7})$ (see Jermyn et al.~2022, ``An Atlas of Convection in Main-Sequence Stars'', in prep).
    Our simulation is as turbulent as possible while also achieving the long-term entrainment of the Ledoux boundary, and is qualitatively in the same regime as stars ($\rm{Pe} \gg 1$, $\rm{Pr} < 1$, $\mS \gg 1$).
    Unfortunately, stars are both more turbulent and have stiffer boundaries than can be simulated with current computational resources.
}
