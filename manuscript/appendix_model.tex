\section{Model \& Initial Conditions}
\label{app:model}
In this work we study the simplest possible system: incompressible, Boussinesq convection with a composition field and a height-varying background radiative conductivity, similar to that used in \citet{fuentes_cumming_2020} and \citet{anders_etal_2022}.
These equations are
\begin{align}
    &\grad\dot\vec{u} = 0 
        \label{eqn:incompressible} \\
    &\partial_t \vec{u} + \vec{u}\dot\grad\vec{u} + \grad \varpi = \left(T - \frac{\mu}{\Ro}\right) \hat{z} + \frac{\Pran}{\mP}\grad^2 \vec{u}
        \label{eqn:momentum}, \\
    &\partial_t T + \vec{u}\dot\grad T + w \grad_{\rm{ad}}  + \grad\dot[-\kappa_{T,0} \grad \overline{T}] = \frac{1}{\mP}\grad^2 T'
        \label{eqn:temperature},
\end{align}
\begin{align}
    &\partial_t \mu + \vec{u}\dot\grad \mu = -\frac{\tau_0}{\mP}\grad^2\bar{\mu} + \frac{\tau}{\mP}\grad^2 \mu'.
        \label{eqn:composition}
\end{align}
Here, $\vec{u}$ is the nondimensional velocity, $T$ is the nondimensional temperature, and $\mu$ is the nondimensional concentration.
Bars (e.g., $\bar{T}$) represent the horizontally-averaged component of a field and primes (e.g., $T'$) denote all fluctuations around that background.
The nondimensional control parameters are
\begin{equation}
\begin{split}
    &\mP = \frac{u_{\rm{ff}} L_{\rm{conv}}}{\kappa_T},\qquad
    \Ro = \frac{|\alpha|\Delta T}{|\beta|\Delta \mu},\qquad\\
    &\Pran = \frac{\nu}{\kappa_T},\qquad
    \tau = \frac{\kappa_\mu}{\kappa_T},\qquad
\end{split}
\end{equation}
where the nondimensional freefall velocity is $\vec{u}_{\rm{ff}} = \sqrt{|\alpha|g L_{\rm{conv}}\Delta T}$ with $g$ the constant gravitational acceleration, $L_{\rm{conv}}$ is the initial depth of the convection zone, $\Delta \mu$ is the composition change across the Ledoux stable region, $\Delta T = L_{\rm{conv}}(\partial_z T_{\rm{rad}} - \partial_z T_{\rm{ad}})$ is the superadiabatic temperature scale of the convection zone, $\alpha$ and $\beta$ are the coefficients of expansion for $T$ and $\mu$, $\nu$ is the viscosity, $\kappa_T$ is the thermal diffusivity, and $\kappa_\mu$ is the compositional diffusivity.
We also specify different values of $\kappa_T = \kappa_{T,0}$ and $\kappa_\mu/\kappa_T = \tau_0$ for the horizontally-averaged component; this allows the radiative gradient to change with height and reduces diffusion on the mean $\mu$ structure to ensure evolution is due to advection.
These equations are described in detail in Sec.~2 of \citet{garaud_2018}, except for the differing diffusivities on the averages and fluctuations.

We define the Ledoux and Schwarzschild discriminants
\begin{equation}
    \yS = \left(\frac{\partial T}{\partial z}\right)_{\rm{rad}} - \left(\frac{\partial T}{\partial z}\right)_{\rm{ad}},\,\,
    \yL = \yS - \Ro^{-1} \frac{\partial \mu}{\partial z},
\end{equation}
and in this nondimensional system the {\brunt} frequency is the negative of the Ledoux discriminant $N^2 = -yL$.

In this work, we study a three-layer model in $z = [0, 3]$,
\begin{align}
    &\left(\frac{\partial T}{\partial z}\right)_{\rm{rad}} = 
    \left(\frac{\partial T}{\partial z}\right)_{\rm{ad}} + 
    \begin{cases}
        -1           & z \leq 2 \\
        10\Ro^{-1}     & z > 2
    \end{cases},
    \label{eqn:initial_T}
    \\
    &\frac{\partial \mu_0}{\partial z} = 
    \begin{cases}
        0        & z \leq 1 \\
        -1       & 1 < z \leq 2 \\
        0        & 2 < z
    \end{cases},
    \label{eqn:initial_mu}
\end{align}
where the intial temperature derivative is $\partial T_0 / \partial z = (\partial T / \partial z)_{\rm{rad}}$ everywhere except between $z = [0.1, 1]$ where it is adiabatic.
We set $(\partial T / \partial z){\rm{ad}} = -1 - 10\Ro^{-1}$.

