% Preamble
%\documentclass[twocolumn, linenumbers]{aastex631}
\documentclass[twocolumn]{aastex631}
\usepackage{natbib}
\usepackage{latexsym}
\usepackage{graphicx}
\usepackage{epsfig}
\usepackage{amssymb}
\usepackage{amsmath}
\usepackage{epstopdf}
\usepackage{hyperref}
\usepackage{xcolor}

%%%% Custom commands
\newcommand{\gradrad}{\ensuremath{\nabla_{\rm{rad}}}}
\newcommand{\gradad}{\ensuremath{\nabla_{\rm{ad}}}}
\newcommand{\justgrad}{\ensuremath{\nabla}}
\newcommand{\delp}{\ensuremath{\delta_{\rm{p}}}}
\newcommand{\Fbot}{\ensuremath{F_{\rm{bot}}}}
\newcommand{\Ftot}{\ensuremath{F_{\rm{tot}}}}
\newcommand{\Frad}{\ensuremath{F_{\rm{rad}}}}
\newcommand{\Fconv}{\ensuremath{F_{\rm{conv}}}}
\newcommand{\Fcz}{\ensuremath{F_{\rm{cz}}}}
\newcommand{\mP}{\ensuremath{\mathcal{P}}}
\newcommand{\mD}{\ensuremath{\mathcal{D}}}
\newcommand{\dP}{\ensuremath{\delta_{\rm{p}}}}
\newcommand{\Lcz}{\ensuremath{L_{\rm{CZ}}}}
\newcommand{\mR}{\ensuremath{\mathcal{R}}}
\newcommand{\mS}{\ensuremath{\mathcal{S}}}
\newcommand\Pran{\ensuremath{\mathrm{Pr}}}
\newcommand{\brunt}{Brunt-V\"{a}is\"{a}l\"{a}}

\newcommand{\angles}[1]{\langle #1 \rangle}
\newcommand{\pd}[1]{\partial_{#1}}
\renewcommand{\vec}[1]{\boldsymbol{#1}}
\newcommand{\M}[1]{\mathbf{#1}}
\renewcommand{\dot}{\vec{\cdot}}
\renewcommand{\bar}[1]{\overline{#1}}
\newcommand{\grad}{\vec{\nabla}}
\newcommand{\cross}{\vec{\times}}
\newcommand{\laplacian}{\nabla^2}

\newcommand{\editone}[1]{\textcolor{orange}{#1}}

%%%% Journal preamble
\received{July 28, 2021}
\revised{October 19, 2021}
\accepted{}
\published{}
\submitjournal{ApJ}

\shorttitle{Convective Penetration}
\shortauthors{Anders et al}


\begin{document}

%%%% Title and Abstract
\title{Schwarzschild or Ledoux: composition gradients are fragile}
\author[0000-0002-3433-4733]{Evan H. Anders}
\affiliation{CIERA, Northwestern University, Evanston IL 60201, USA}
\author[0000-0001-5048-9973]{Adam S. Jermyn}
\affiliation{Center for Computational Astrophysics, Flatiron Institute, New York, NY 10010, USA}
\author[0000-0002-7635-9728]{Daniel Lecoanet}
\affiliation{CIERA, Northwestern University, Evanston IL 60201, USA}
\affiliation{Department of Engineering Sciences and Applied Mathematics, Northwestern University, Evanston IL 60208, USA}
%\author[0000-0001-8935-219X]{Adrian E. Fraser}
\author[0000-0003-4323-2082]{Adrian E. Fraser}
\affiliation{University of California, Santa Cruz, Santa Cruz, California 95064, U.S.A}
\author{Imogen G. Cresswell}
\affiliation{Department Astrophysical and Planetary Sciences \& LASP, University of Colorado, Boulder, CO 80309, USA}

\correspondingauthor{Evan H. Anders}
\email{evan.anders@northwestern.edu}

\begin{abstract}
    This will be an abstract.
\end{abstract}
\keywords{UAT keywords}



%%%% Body of paper
\section{Introduction}
\label{sec:introduction}


\section{Central Result: }
%\label{sec:central_result}
%\begin{figure}[t]
%\centering
%\includegraphics[width=\columnwidth]{grad_profiles.pdf}
%\caption{
%Horizontally- and temporally-averaged profiles of the thermodynamic gradients from the simulation in Fig.~\ref{fig:vertical_dynamics_panels}.
%We plot $\justgrad$ (green) compared to $\gradad$ (purple, a constant) and $\gradrad$ (orange); note the extended penetration zone (PZ) where $\justgrad \approx \gradad > \gradrad$.
%    \editone{
%        The dashed vertical line denotes the Schwarzschild boundary of the convection zone (CZ), the solid vertical line denotes the bottom of the radiative zone (RZ), and the greyed region denotes the PZ-RZ boundary layer.
%    }
%\label{fig:grad_profiles}
%}
%\end{figure}



\section{Theory}
\label{sec:theory}

%\begin{align}
%&\grad\dot\vec{u} = 0 
%\label{eqn:incompressible} \\
%&\partial_t \vec{u} + \vec{u}\dot\grad\vec{u} = -\frac{1}{\rho_0}\grad p + \frac{\rho_1}{\rho_0}\vec{g} + \nu\grad^2 \vec{u} 
%\label{eqn:momentum} \\
%&\partial_t T + \vec{u}\dot\grad T + w \gradad + \grad\dot[-k \grad \overline{T}] = \chi\grad^2 T' + Q
%\label{eqn:temperature} \\
%&\frac{\rho_1}{\rho_0} = -|\alpha| T.
%\label{eqn:boussinesq}
%\end{align}

\section{Simulation Details}
\label{sec:sim_details}

\section{Results}
\label{sec:results}


\section{Discussion}
\label{sec:discussion}


\begin{acknowledgments}
We thank Dominic Bowman, Meridith Joyce, ETC ETC ETC.
EHA is funded as a CIERA Postdoctoral fellow and would like to thank CIERA and Northwestern University. 
We acknowledge the hospitality of Nordita during the program ``The Shifting Paradigm of Stellar Convection: From Mixing Length Concepts to Realistic Turbulence Modelling," where the groundwork for this paper was set.
This work was supported by NASA HTMS grant 80NSSC20K1280 and NASA SSW grant 80NSSC19K0026.
Computations were conducted with support from the NASA High End Computing (HEC) Program through the NASA Advanced Supercomputing (NAS) Division at Ames Research Center on Pleiades with allocation GID s2276.
The Flatiron Institute is supported by the Simons Foundation.
\end{acknowledgments}


\appendix

\section{Table of simulation parameters}
\label{app:simulation_table}
Input parameters and summary statistics of the simulations presented in this work are shown in Table~\ref{table:simulation_info}.

%\begin{deluxetable*}{c c c c c c c c c c}
%\tabletypesize{\footnotesize}
%\caption{Table of simulation information.
%\label{table:simulation_info}
%}
%\tablehead{
%\colhead{Type} 	& \colhead{$\mP$}	& \colhead{$\mS$}	& \colhead{$\mR$}	& \colhead{$nx \times ny \times nz$}	&	\colhead{$t_{\rm{sim}}$}	&	\colhead{$(\delta_{0.1}, \delta_{0.5}, \delta_{0.9})$}	& \colhead{$f$}	& \colhead{$\xi$}	& \colhead{$\angles{u}$}
%}
%\startdata
%\multicolumn{3}{l}{\textbf{``Standard timestepping'' simulations}}\\
%D & $1.0$ & $ 10^{3}$ & $ 4.0 \cdot 10^{2}$ &  64x64x256   & $ 12347$ & (0.078, 0.112, 0.136) &  0.810 &  0.682 &  0.618 \\
%D & $ 2.0$ & $ 10^{3}$ & $ 4.0 \cdot 10^{2}$ &  64x64x256   & $ 32057$ & (0.200, 0.230, 0.254) &  0.749 &  0.601 &  0.639 \\
%D & $ 4.0$ & $ 10^{3}$ & $ 4.0 \cdot 10^{2}$ &  64x64x256   & $ 66557$ & (0.445, 0.472, 0.496) &  0.668 &  0.562 &  0.619 \\
%\hline
%\multicolumn{3}{l}{\textbf{``Accelerated Evolution'' simulations}}\\
%D & $ 4.0$ & $ 10^{2}$ & $ 4.0 \cdot 10^{2}$ &  64x64x256   & $ 5000 $ & (0.377, 0.505, 0.581) &  0.654 &  0.526 &  0.617 \\
%D & $ 4.0$ & $3.0 \cdot 10^{2}$ & $ 4.0 \cdot 10^{2}$ &  64x64x256   & $ 5000 $ & (0.420, 0.477, 0.514) &  0.663 &  0.551 &  0.618 \\
%D & $ 10^{-1}$ & $ 10^{3}$ & $ 4.0 \cdot 10^{2}$ &  64x64x256   & $ 4561 $ & (0.017, 0.042, 0.069) &  0.831 &  0.769 &  0.588 \\
%D & $ 3.0 \cdot 10^{-1}$ & $ 10^{3}$ & $ 4.0 \cdot 10^{2}$ &  64x64x256   & $ 4681 $ & (0.030, 0.064, 0.092) &  0.814 &  0.804 &  0.620 \\
%D & $1.0$ & $ 10^{3}$ & $ 4.0 \cdot 10^{2}$ &  64x64x256   & $ 3000 $ & (0.082, 0.116, 0.140) &  0.804 &  0.690 &  0.624 \\
%D & $ 2.0$ & $ 10^{3}$ & $ 4.0 \cdot 10^{2}$ &  64x64x256   & $ 5000 $ & (0.199, 0.228, 0.252) &  0.750 &  0.597 &  0.638 \\
%D & $ 4.0$ & $ 10^{3}$ & $ 2.5 \cdot 10^{1}$ &  16x16x256   & $ 3000 $ & (0.321, 0.379, 0.437) &  0.772 &  0.274 &  0.343 \\
%D & $ 4.0$ & $ 10^{3}$ & $ 5.0 \cdot 10^{1}$ &  32x32x256   & $ 3000 $ & (0.398, 0.442, 0.487) &  0.732 &  0.358 &  0.423 \\
%D & $ 4.0$ & $ 10^{3}$ & $ 10^{2}$ &  32x32x256   & $ 3000 $ & (0.469, 0.503, 0.534) &  0.672 &  0.464 &  0.484 \\
%D & $ 4.0$ & $ 10^{3}$ & $ 2.0 \cdot 10^{2}$ &  64x64x256   & $ 3000 $ & (0.485, 0.515, 0.542) &  0.648 &  0.546 &  0.548 \\
%D & $ 4.0$ & $ 10^{3}$ & $ 4.0 \cdot 10^{2}$ &  64x64x256   & $ 5000 $ & (0.452, 0.480, 0.505) &  0.667 &  0.553 &  0.617 \\
%D & $ 4.0$ & $ 10^{3}$ & $ 8.0 \cdot 10^{2}$ &  128x128x256 & $ 3000 $ & (0.407, 0.434, 0.455) &  0.689 &  0.566 &  0.678 \\
%D & $ 4.0$ & $ 10^{3}$ & $ 1.6 \cdot 10^{3}$ &  128x128x256 & $ 3000 $ & (0.366, 0.397, 0.419) &  0.709 &  0.574 &  0.720 \\
%D & $ 4.0$ & $ 10^{3}$ & $ 3.2 \cdot 10^{3}$ &  256x256x256 & $ 3235 $ & (0.321, 0.358, 0.381) &  0.723 &  0.605 &  0.746 \\
%D & $ 4.0$ & $ 10^{3}$ & $ 6.4 \cdot 10^{3}$ &  384x384x384 & $ 414  $ & (0.277, 0.315, 0.335) &  0.744 &  0.605 &  0.757 \\
%D & $ 6.0$ & $ 10^{3}$ & $ 4.0 \cdot 10^{2}$ &  64x64x256   & $ 6000 $ & (0.620, 0.647, 0.667) &  0.635 &  0.532 &  0.597 \\
%D & $ 8.0$ & $ 10^{3}$ & $ 4.0 \cdot 10^{2}$ &  128x128x512 & $ 4357 $ & (0.732, 0.759, 0.779) &  0.640 &  0.481 &  0.592 \\
%D & $ 10^{1}$ & $ 10^{3}$ & $ 4.0 \cdot 10^{2}$ &  128x128x512 & $ 4226 $ & (0.858, 0.885, 0.904) &  0.630 &  0.453 &  0.587 \\
%D & $ 4.0$ & $3.0 \cdot 10^{3}$ & $ 4.0 \cdot 10^{2}$ &  64x64x512   & $ 1170 $ & (0.437, 0.454, 0.469) &  0.672 &  0.581 &  0.619 \\
%D/SF & $ 4.0$ & $ 10^{3}$ & $ 5.0 \cdot 10^{1}$ &  32x32x256   & $ 5000 $ & (0.435, 0.477, 0.516) &  0.680 &  0.418 &  0.505 \\
%D/SF & $ 4.0$ & $ 10^{3}$ & $ 10^{2}$ &  32x32x256   & $ 5000 $ & (0.482, 0.516, 0.547) &  0.638 &  0.543 &  0.573 \\
%D/SF & $ 4.0$ & $ 10^{3}$ & $ 2.0 \cdot 10^{2}$ &  64x64x256   & $ 5000 $ & (0.490, 0.520, 0.547) &  0.634 &  0.589 &  0.640 \\
%D/SF & $ 4.0$ & $ 10^{3}$ & $ 4.0 \cdot 10^{2}$ &  64x64x256   & $ 8000 $ & (0.474, 0.502, 0.531) &  0.651 &  0.588 &  0.693 \\
%D/SF & $ 4.0$ & $ 10^{3}$ & $ 8.0 \cdot 10^{2}$ &  128x128x256 & $ 5000 $ & (0.410, 0.437, 0.461) &  0.683 &  0.587 &  0.732 \\
%D/SF & $ 4.0$ & $ 10^{3}$ & $ 1.6 \cdot 10^{3}$ &  128x128x256 & $ 5710 $ & (0.368, 0.400, 0.426) &  0.703 &  0.590 &  0.758 \\
%D/SF & $ 4.0$ & $ 10^{3}$ & $ 3.2 \cdot 10^{3}$ &  256x256x256 & $ 3917 $ & (0.320, 0.357, 0.388) &  0.725 &  0.595 &  0.772 \\
%L & $ 10^{-2}$ & $ 10^{3}$ & $ 8.0 \cdot 10^{2}$ &  128x128x256 & $ 1139 $ & (0.017, 0.030, 0.051) &  0.873 &  0.783 &  0.445 \\
%L & $ 3.0 \cdot 10^{-2}$ & $ 10^{3}$ & $ 8.0 \cdot 10^{2}$ &  128x128x256 & $ 929  $ & (0.020, 0.044, 0.070) &  0.863 &  0.782 &  0.448 \\
%L & $ 10^{-1}$ & $ 10^{3}$ & $ 8.0 \cdot 10^{2}$ &  128x128x256 & $ 1142 $ & (0.081, 0.076, 0.102) &  0.848 &  0.725 &  0.450 \\
%L & $ 3.0 \cdot 10^{-1}$ & $ 10^{3}$ & $ 8.0 \cdot 10^{2}$ &  128x128x256 & $ 1109 $ & (0.076, 0.129, 0.157) &  0.825 &  0.655 &  0.451 \\
%L & $1.0$ & $ 10^{3}$ & $ 8.0 \cdot 10^{2}$ &  128x128x256 & $ 3000 $ & (0.182, 0.225, 0.251) &  0.787 &  0.599 &  0.442 \\
%L & $ 2.0$ & $ 10^{3}$ & $ 8.0 \cdot 10^{2}$ &  128x128x256 & $ 3000 $ & (0.278, 0.315, 0.340) &  0.759 &  0.570 &  0.436 \\
%L & $ 4.0$ & $ 10^{3}$ & $ 8.0 \cdot 10^{2}$ &  128x128x256 & $ 10000$ & (0.399, 0.431, 0.455) &  0.737 &  0.518 &  0.428 \\
%L & $ 8.0$ & $ 10^{3}$ & $ 8.0 \cdot 10^{2}$ &  128x128x256 & $ 5000 $ & (0.519, 0.545, 0.562) &  0.718 &  0.484 &  0.421 \\
%L & $ 1.6 \cdot 10^{1}$ & $ 10^{3}$ & $ 8.0 \cdot 10^{2}$ &  128x128x256 & $ 8000 $ & (0.687, 0.709, 0.723) &  0.700 &  0.442 &  0.417 \\
%\enddata                                                   	
%\tablecomments{                          
%Simulation type is specified as ``D'' for discontinuous/Case I or ``L'' for linear/Case II.
%``D/SF'' simulations have stress-free boundary conditions.
%Input control parameters are listed for each simulation: the penetration parameter $\mP$, stiffness $\mS$, and freefall Reynolds number $\mR$.
%We also note the coefficient resolution (Chebyshev coefficients $nz$ and Fourier coeficients $nx$, $ny$).
%We report the number of freefall time units each simulation was run for $t_{\rm{sim}}$.
%Time-averaged values of the departure heights ($\delta_{0.1}$, $\delta_{0.5}$, $\delta_{0.9}$), the dissipation fraction $f$, and the dissipation fall-off $\xi$, as well as the average convection zone velocity $\angles{u}$ are reported.
%We take these time averages over the final 1000 freefall times or half of the simulation, whichever is shorter.
%}                                                          	
%\end{deluxetable*}                                        	
%                                                          	

\newpage
\,
\newpage
\bibliographystyle{aasjournal}
\bibliography{biblio}
\end{document}
