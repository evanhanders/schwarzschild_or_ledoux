% Preamble
\documentclass[twocolumn, linenumbers, twocolappendix]{aastex631}
%\documentclass[twocolumn]{aastex631}
\usepackage{natbib}
\usepackage{latexsym}
\usepackage{graphicx}
\usepackage{epsfig}
\usepackage{amssymb}
\usepackage{amsmath}
\usepackage{epstopdf}
\usepackage{hyperref}
\usepackage{xcolor}

%%%% Custom commands
\newcommand{\yL}{\ensuremath{\mathcal{Y}_{\rm{L}}}}
\newcommand{\yS}{\ensuremath{\mathcal{Y}_{\rm{S}}}}
\newcommand{\justgrad}{\ensuremath{\nabla}}
\newcommand{\gradrad}{\ensuremath{\nabla_{\rm{rad}}}}
\newcommand{\gradad}{\ensuremath{\nabla_{\rm{ad}}}}
\newcommand{\gradC}{\ensuremath{\nabla_{\mathrm{C}}}}
\newcommand{\gradmu}{\ensuremath{\nabla_{\mu}}}
\newcommand{\gradL}{\ensuremath{\nabla_{\mathrm{L}}}}
\newcommand{\gradT}{\ensuremath{\nabla_{\mathrm{T}}}}
\newcommand{\Ro}{\ensuremath{\mathrm{R}_{0}}}
\newcommand{\delp}{\ensuremath{\delta_{\rm{p}}}}
\newcommand{\Fbot}{\ensuremath{F_{\rm{bot}}}}
\newcommand{\Ftot}{\ensuremath{F_{\rm{tot}}}}
\newcommand{\Frad}{\ensuremath{F_{\rm{rad}}}}
\newcommand{\Fconv}{\ensuremath{F_{\rm{conv}}}}
\newcommand{\Fcz}{\ensuremath{F_{\rm{cz}}}}
\newcommand{\mP}{\ensuremath{\mathcal{P}}}
\newcommand{\mD}{\ensuremath{\mathcal{D}}}
\newcommand{\dP}{\ensuremath{\delta_{\rm{p}}}}
\newcommand{\Lcz}{\ensuremath{L_{\rm{CZ}}}}
\newcommand{\mR}{\ensuremath{\mathcal{R}}}
\newcommand{\mS}{\ensuremath{\mathcal{S}}}
\newcommand\Pran{\ensuremath{\mathrm{Pr}}}
\newcommand{\brunt}{{Brunt-V\"{a}is\"{a}l\"{a}}}

\newcommand{\angles}[1]{\langle #1 \rangle}
\newcommand{\pd}[1]{\partial_{#1}}
\renewcommand{\vec}[1]{\boldsymbol{#1}}
\newcommand{\M}[1]{\mathbf{#1}}
\renewcommand{\dot}{\vec{\cdot}}
\renewcommand{\bar}[1]{\overline{#1}}
\newcommand{\grad}{\vec{\nabla}}
\newcommand{\cross}{\vec{\times}}
\newcommand{\laplacian}{\nabla^2}

\newcommand{\editone}[1]{\textcolor{orange}{#1}}

%%%% Journal preamble
\received{July 28, 2021}
\revised{October 19, 2021}
\accepted{}
\published{}
\submitjournal{ApJ}

\shorttitle{Schwarzschild or Ledoux}
\shortauthors{Anders et al}


\begin{document}

%%%% Title and Abstract
\title{Schwarzschild and Ledoux are equivalent on evolutionary timescales}
\author[0000-0002-3433-4733]{Evan H. Anders}
\affiliation{CIERA, Northwestern University, Evanston IL 60201, USA}
\affiliation{Kavli Institute for Theoretical Physics, University of California, Santa Barbara, CA 93106, USA}
\author[0000-0001-5048-9973]{Adam S. Jermyn}
\affiliation{Center for Computational Astrophysics, Flatiron Institute, New York, NY 10010, USA}
\affiliation{Kavli Institute for Theoretical Physics, University of California, Santa Barbara, CA 93106, USA}
\author[0000-0002-7635-9728]{Daniel Lecoanet}
\affiliation{CIERA, Northwestern University, Evanston IL 60201, USA}
\affiliation{Department of Engineering Sciences and Applied Mathematics, Northwestern University, Evanston IL 60208, USA}
\affiliation{Kavli Institute for Theoretical Physics, University of California, Santa Barbara, CA 93106, USA}
%\author[0000-0001-8935-219X]{Adrian E. Fraser}
\author[0000-0003-4323-2082]{Adrian E. Fraser}
\affiliation{University of California, Santa Cruz, Santa Cruz, California 95064, U.S.A}
\affiliation{Kavli Institute for Theoretical Physics, University of California, Santa Barbara, CA 93106, USA}
\author[0000-0002-4538-7320]{Imogen G. Cresswell}
\affiliation{Department Astrophysical and Planetary Sciences \& LASP, University of Colorado, Boulder, CO 80309, USA}
\affiliation{Kavli Institute for Theoretical Physics, University of California, Santa Barbara, CA 93106, USA}
\author[0000-0003-2124-9764]{J. R. Fuentes}
\affiliation{Department of Physics and McGill Space Institute, McGill University, 3600 rue University, Montreal, QC H3A 2T8, Canada}

\correspondingauthor{Evan H. Anders}
\email{evan.anders@northwestern.edu}

\begin{abstract}
    In stars, convective boundaries are set by either the Schwarzschild or Ledoux criterion, but there is no consensus regarding which criterion to use.
    In this letter, we present a 3D hydrodynamical simulation of a convection zone whose boundary is thermally unstable but compositionally stable.
    Therefore, the convective boundary is Schwarzschild unstable but Ledoux stable.
    Over many convective overturn timescales, entrainment makes the convection zone grow.
    The convective boundary stops advancing after it becomes stable by the Schwarzschild criterion.
    This work demonstrates using 3D simulations that the Ledoux criterion \emph{instantaneously} describes the boundary of a convection zone, but that Ledoux-stable boundaries are fragile unless they are also Schwarzschild stable.
    Therefore, the Schwarzschild stability criterion describes the size of a convection zone, except during short-lived evolutionary stages in which convection does not reach statistical equilibrium.
\end{abstract}
\keywords{Stellar convection zones (301), Stellar physics (1621); Stellar evolutionary models (2046)}



%%%% Body of paper

\section{Introduction}
\label{sec:introduction}

\begin{itemize}
\item Observations tell us that we don't understand the mixing near convective boundaries.
\item One fundamental question is: what stability criterion should be used? Define Schwarzschild and Ledoux boundaries.
\item Naive implementation e.g., in \citet{mesa2} run into problems in the context of MLT logic.
\item These naive implementations also are at odds with simulations: high entrainment rates (citations), and entrainment theories suggest that ledoux-stable but schwarzschild-unstable regions should be fully mixed \citep{fuentes_cumming_2020}, more citations.
\item Furthermore, \citet{gabriel_etal_2014} point out that a naive use of the Ledoux criterion is logically inconsistent within the framework of mixing length theory.
\item New algorithms \citep{mesa4, mesa5} have been developed which handle boundaries self-consistently and essentially make Schwarzschild and Ledoux criterion the same.
\item This theoretical discussion has occurred in the context of theoretical arguments and 1D models, but convection is a three-dimensional process.
\item Fundamental, simple experiments which demonstrate the efficacy of these new algorithms, and whether convection prefers the Schwarzschild or Ledoux stability criterion, are lacking.
\item In this work, we present a simple three-dimensional experiment that demonstrates that regions that are stable by the Ledoux criterion are fragile, and that, at late times, the Ledoux and Schwarzschild criterion both find the same edge of the convection zone.
\end{itemize}



\begin{figure*}[t!]
\centering
\includegraphics[width=\textwidth]{dynamics_figure.pdf}
\caption{
    Volume renderings of the simulation composition field $\mu$ at early (left) and late (right) times.
    The change in color from white at the top of the box to dark purple at the top of the convection zone denotes a stable composition gradient.
    The convection zone is mostly well-mixed, so we expand the colorbar scaling there; black represents entrained low-composition fluid being mixed into the yellow high-composition convection zone.
    The orange and purple horizontal lines respectively denote the Ledoux and Schwarzschild boundaries.
    The simulation domain spans $z = [0, 3]$, but we only plot $z = [0, 2.5]$ here.
    {\color{blue} Note: this is currently evolving further.}
\label{fig:dynamics}
}
\end{figure*}

\section{Theory}
\label{sec:theory}

\begin{itemize}
\item The stability of a convective region can instantaneously be determined using the Schwarzschild criterion,
\begin{equation}
    y_S = \gradrad - \gradad,
\end{equation}
or the Ledoux criterion,
\begin{equation}
y_L = \gradrad - \gradL,
\end{equation}
with the unstable convection zone contained within the region where $y < 0$ \citep[see e.g., section 2 of][]{mesa4}.
\item The \emph{instantaneous} stability changes over the course of many convective freefall times, and stellar evolution timesteps generally span many convective overturn times.
\item The goal of this work is to find if $y_S$ and $y_L$ evolve toward the same value, and whether they evolve toward the initial boundary as determined by $y_S$ or $y_L$.
\item We will study a numerical simulation with three layers: a convective layer $y_s < 0$ and $y_L < 0$, a ``semiconvective'' \citep[see section 4 of][]{mesa2} \citep{salaris_cassisi_2017} layer with $y_S < 0$ but $y_L > 0$ \citep[which is stable to ODDC][]{garaud_2018}, and a fully stable layer with $y_L > 0$ and $y_S > 0$.
\item While ``semiconvective'' layers are often unstable to doubly-diffusive instabilities in stellar regimes \citep{moore_garaud_2016}, our purpose in this work is to demonstrate that regardless of whether these layers are doubly-diffusive unstable, they are entrained and destroyed by neighboring convective zones.
\end{itemize}



\begin{figure*}[t]
\centering
\includegraphics[width=\textwidth]{fig2_profiles.pdf}
\caption{
    Horizontally-averaged profiles are shown for the composition (top), the discriminants $\yS$ and $\yL$ (middle, Eqns.~\ref{eqn:yS} \& \ref{eqn:yL}), and the \brunt$\,$ frequency $N^2 = -\yL$ and the square convective frequency $f_{\rm{conv}}^2$ (bottom, Eqn.~\ref{eqn:fconv2}).
    Positive and negative values are respectively solid and dashed lines.
    We show the initial (left) and evolved (right, time-averaged over 100 convective overturn times) states.
    The background color is orange in CZs, green in SZs, and purple in RZs per Section ~\ref{sec:theory}.
    The lightly hashed background region in the evolved RZ is the mechanical overshoot zone.
    {\color{blue} Note: this data is taken from a less turbulent run than fig 1; it'll be updated when the fig 1 run finishes.}
\label{fig:profiles}
}
\end{figure*}



\section{Results}
\label{sec:result}


Volume visualizations of simulation dynamics are shown near the initial state (left) and evolved state (right) in Fig.~\ref{fig:dynamics}.
Buoyancy perturbations normalized by the vertical profile of buoyancy standard deviations are shown in the top two panels.
Vertical velocity is shown in the bottom two panels.
In the initial state, convection occurs in the bottom $\sim1/3$ of the simulation domain; the middle $\sim1/3$ of the domain is stabilized by a composition gradient, and the top $\sim1/3$ is stabilized by a thermal gradient.
The convection excites gravity waves in the stable layers.
The \brunt$\,$ frequency is higher by a factor of 10 in the thermal layer than in the semiconvection layer, so the vertical velocity signature of motions there is smaller than in the semiconvection layer.
Describe overhoot.

The most obvious difference between the panels on the left and the right is that the convection zone has grown in size from $\sim 1/3$ of the simulation domain to $\sim 2/3$ of the simulation domain.
Through continuous overshoot, convection entrained stable, low-composition fluid from the upper region into the convection zone.
This process eroded the composition gradient until the Schwarzschild and Ledoux boundaries of the convection zone were identical.
In other words, the \emph{thermal} stability of the upper zone is sufficient to halt expansion of the convection zone via entrainment, but compositional stability is not.
We see negligible convective penetration (mixing of the bouyancy or entropy profile beyond the sign change in $mathcal{Y}$), but this is expected and part of our experimental design (see appendix).

Figure \ref{fig:profiles} displays vertical profiles that have been averaged horizontally and in time.
Profiles on the left show initial conditions, while profiles on the right show the evolved state.
We show the composition (top panels), the Schwarzschild and Ledoux discriminants (middle panels), and the square \brunt$\,$ and convective frequencies.

In the initial state, we see that the composition is uniform in the CZ ($z \leq 1$) and RZ ($z \geq 2$), but varies linearly in $z \in [1, 2]$ and provides stability.
We also see that the sign change in $\yL$ occurs at $z \sim 1$ while that in $\yS$ occurs at $z \sim 2$.
Finally, we see that $f_{\rm{conv}} = 0$ because we initialize the simulation without any convective velocity.
However, the \brunt$\,$ frequency $N^2$ is negative in a boundary layer at the base of the CZ which drives the instability, and $N^2$ is stable above $z = 1$ (and is more stable by a factor of 10 above $z \sim 2$).

The final state (right) is attained after convection entrains and mixes through the initial composition gradient.
We see that the composition profile (top) is constant in the convection zone, and approximates a step function above the CZ at the top of the overshoot zone. (TODO: Add overshoot to this figure).
In this evolved state, the sign changes in the discriminants $\yL$ and $\yS$ coincide (middle panel).
In the bottom panel, we see that the convective frequency is roughly constant, and see that $N^2 \lesssim 0$ in the bulk CZ.
We can compute the ``stiffness'' $\mS = N^2 / f_{\rm{conv}}^2$ of the radiative-convective boundary by comparing the average CZ value of $f_{\rm{conv}}^2 \sim 10^{-2}$ to the RZ value of $N^2 \sim 10^2$, so $\mS \approx 10^4$.
Boundaries with a low stiffness $\mS \lesssim 10$ easily deform in the presence of convective flows, but convective boundaries in stars often have $\mS \gtrsim 10^6$.
The value of $\mS$ achieved in these simulations is therefore in the right regime to tell us about stars, but these simulations still exhibit more mechanical overshoot than we would expect stars to.


Finally, In figure \ref{fig:kippenhahn}, we plot a Kippenhahn-like diagram of the simulation.
The CZ is shown in orange and is the region below the sign change of both $\mY$ and $\mS$.
The semiconvection zone is shown in green and is the region below the sign changes of $\mY$ and $\mS$.
The RZ is shown in purple and is the region above the sign change of both $\mY$ and $\mS$.
Convection overshoot roughly above the $\yL = 0$ line up to the black line, denoted by a hashed region.
The height of the black line traces out the region where the vertical profile of the convective kinetic energy falls below 10\% of its value in the bulk CZ; this line roughly coincides with the extremum of the composition gradient through the simulation evolution.
Importantly, while the orange line that traces out $\yL = 0$ and the green line tracing out $\yS = 0$ start at different heights, 3D convective motions make these lines converge on long timescales.




\begin{figure}[t!]
\centering
\includegraphics[width=\columnwidth]{kippenhahn.pdf}
\caption{
    A kippenhahn-like diagram of the simulation evolution.
    The $y$-axis is simulation height and the $x$-axis is simulation time.
    The orange line denotes the Ledoux boundary ($\yL = 0$); the CZ is below this and is colored orange.
    The purple line denotes the Schwarzschild boundary ($\yS = 0$); the RZ is above this and is colored purple.
    The semiconvective region between these boundaries is colored green.
    The black line denotes the top of the hashed overshoot zone.
    The simulation has an ``entrainment phase'' while the CZ expands, and a pure ``overshoot phase'' where the convective boundary does not advance.
    {\color{blue} Note: this data is taken from a less turbulent run than fig 1; it'll be updated when the fig 1 run finishes.}
\label{fig:kippenhahn}
}
\end{figure}


\newpage
\,\newpage
\section{Conclusions \& Discussion}
\label{sec:conclusions}

In this letter, we present 3D simulations of a convection zone and its boundary.
The initial boundary is compositionally stable but weakly thermally unstable (Ledoux stable but Schwarzschild unstable).
Entrainment causes the convective boundary to advance until the Ledoux and Schwarzschild criterion agree upon the location of the convective boundary.

These simulations demonstrate that the Ledoux criterion properly defines the \emph{instantaneous} criterion for the boundary of a convection zone.
However, when the evolutionary timescale $t_{\rm{evolution}} \gg t_{\rm{conv}}$, the convective overturn timescale, the Schwarzschild criterion provides the best description of the steady-state boundary of the convection zone.
Our 3D dynamical simulations support the claim that ``logically consistent'' implementations of mixing length theory \citep{gabriel_etal_2014, mesa4, mesa5} must set the Schwarzschild discriminant $\yS = 0$ at the convective boundary.
This suggests that the MESA software instrument's modern ``convective pre-mixing'' (CPM) algorithm should properly find the boundary of most convection zones.
Put differently, our simulations suggest that 1D stellar evolution models should not produce different answers when using the Schwarzschild or Ledoux criterion for convective stability when $t_{\rm{evolution}} \gg t_{\rm{conv}}$.

We note briefly that many Ledoux-stable but Schwarzschild-unstable regions in stars are unstable to overstable doubly-diffusive convection (ODDC).
ODDC generally mixes more quickly than the entrainment studied here, and has been studied extensively in local simulations \citep{mirouh_etal_2012, wood_etal_2013, xie_etal_2017}; see \citet{garaud_2018} for a nice review.
ODDC has been applied in 1D stellar evolution models to the regions near main sequence stellar convective cores in \citet{moore_garaud_2016}.
They find rapid mixing of ledoux-stable but schwarzschild-unstable regions, and ODDC formulations should should be widely included in stellar models.

For stages in stellar evolution where $t_{\rm{conv}} \sim t_{\rm{evolution}}$, implementations of time-dependent convection (TDC, CITE) should be employed to properly capture convective dynamics and the advancement of convective boundaries.
The advancement of convective boundaries in TDC implementations should be informed by time-dependent theories and simulations of the motion of convective boundaries \citep[e.g.,][]{turner_1968, fuentes_cumming_2020}.

The purpose of this study was to understand how the root of the discriminant $\yL$ evolves over time, and whether it coincides with the root of $\yS$ at late times.
While there is interesting behavior near the boundary beyond that point (e.g., mechanical convective overshoot), a detailed analysis of that phenomenon is beyond the scope of this work.
We furthermore constructed the simulations in this work to have a small penetration parameter $\mathcal{P}$ \citep{anders_et_al_2021} and we see negliglble convective penetration in our simulations.
Finally, in our simulations, the radiative conductivity is independent of the magnitude of the composition $\mu$, but this is not the case in stars.
Since the radiative conductivity sets the location of the Schwarzschild boundary, including these effects would change the exact location of our final convective boundary, but would not change the fundamental takeaways of this work.

In summary, we find that the Schwarzschild criterion provides the location of the convective boundary in a statistically stationary state; in this final state, the Ledoux and Schwarzschild criteria are degenerate.


\begin{acknowledgments}
We thank Meridith Joyce, Anne Thoul, Dominic Bowman, Jared Goldberg, Tim Cunningham, Falk Herwig, Kyle Augustson, (OTHERS?) for useful discussions which helped improve our understanding.
EHA is funded as a CIERA Postdoctoral fellow and would like to thank CIERA and Northwestern University. 
This research was supported in part by the National Science Foundation under Grant No. PHY-1748958, and we acknowledge the hospitality of KITP during the Probes of Transport in Stars Program.
Computations were conducted with support from the NASA High End Computing (HEC) Program through the NASA Advanced Supercomputing (NAS) Division at Ames Research Center on Pleiades with allocation GID s2276.
The Flatiron Institute is supported by the Simons Foundation.
\end{acknowledgments}


\appendix

\section{Model \& Initial Conditions}
\label{app:model}
In this work we study the simplest possible system: incompressible, Boussinesq convection with a composition field and a height-varying background radiative conductivity, similar to that used in \citet{fuentes_cumming_2020, anders_etal_2022}.
These equations are
\begin{align}
    &\grad\dot\vec{u} = 0
        \label{eqn:incompressible_dimensional}, \\
    &\partial_t \vec{u}\dot\grad\vec{u} = -\frac{1}{\rho_0}\grad p + \frac{\rho_1}{\rho_0}\vec{g} + \nu\grad^2\vec{u}
        \label{eqn:momentum_dimensional}, \\
    &\partial_t T + \vec{u}\dot\grad T + w\gradad + \grad\dot[-\kappa_{T,0}\grad\bar{T}] = \kappa_T\grad^2 T'
        \label{eqn:temperature_dimensional}, \\
    &\partial_t C + \vec{u}\dot\grad C = \kappa_{C,0}\grad^2\bar{C} + \kappa_C\grad^2 C'
        \label{eqn:composition_dimensional}, \\
    &\frac{\rho_1}{\rho_0} = -|\alpha|T + |\beta|C
        \label{eqn:boussinesq}.
\end{align}
Here, $\vec{u}$ is the vector velocity, $T$ is the temperature, $C$ is the composition, $\rho_0$ is the constant background density, $\p$ is the kinematic pressure which enforces Eqn.~\ref{eqn:incompressible_dimensional}, $\rho_1$ are density fluctuations which act only on the buoyant term, and $\alpha$ and $\beta$ are the thermal and compositional expansion coefficients, and $\gradad$ is the adiabatic gradient.
Diffusive terms are controlled by the kinematic viscosity $\nu$, as well as the thermal diffusivity $\kappa_T$ and compositional diffusivity $\kappa_C$.
On the horizontally-invariant ($n_x = 0$ and $n_y = 0$) mode, we use a height-depended thermal diffusion coefficient $\kappa_{T,0}$ (which allows $\gradrad$ to vary with height) and a lower compositional diffusivity $\kappa_{C,0} < \kappa_C$ to ensure that the evolution of the mean composition profile is due to advection rather than diffusion.

We nondimensionalize Eqns.~\ref{eqn:incompressible_dimensional}-\ref{eqn:boussinesq} on the length scale of the initial Schwarzschild-unstable convection zone $L_s$, the timescale of freefall across that convection zone 

\begin{equation}
    \tau_{\rm{ff}} = \left(\frac{L_s}{|\alpha| g \Delta T}\right)^{1/2},
\end{equation}
and the temperature scale set by the temperature gradient at the bottom boundary $\Delta T = L_s(\partial_z T)_{\rm{bot}}$; mass is nondimensionalized so that the freefall ram pressure $\rho_0(L_s/\tau_{\rm{ff}})^2 = 1$, and composition is nondimensionalized so that its value is initially 1 in the convection zone and zero in the Schwarzschild-stable radiative zone.
This nondimensionalization is
\begin{equation}
\begin{split}
    &T^* = (\Delta T)T = (L_s [\partial_z T]_{\rm{bot}}) tQ_0 \tau_{\rm{ff}} T,\qquad
    C^* = (\Delta C)C,\qquad
    \partial_{t^*} = \tau_{\rm{ff}}^{-1}\partial_t,\qquad\,\,\,
    \grad^* = L_s^{-1} \grad,\,\,\,\,\,
\\
    &\vec{u}^* = u_{\rm{ff}}\vec{u} = \frac{L_s}{\tau_{\rm{ff}}} \vec{u}, \qquad\qquad
    p^* = \rho_0 u_{\rm{ff}}^2\varpi,\qquad
    \kappa_T^* = (L_s^2 \tau_{\rm{ff}}^{-1})$\kappa_T$,\qquad
    \kappa_C^* = (L_s^2 \tau_{\rm{ff}}^{-1})$\kappa_C$.
\end{split}
\end{equation}
For convenience, here we define quantities with $*$ (e.g., $T^*$) as being the ``dimensionful'' quantities of Eqns.~\ref{eqn:incompressible}-\ref{eqn:boussinesq}.
Henceforth, quantities without $*$ (e.g., $T$) are dimensionless.
Within this nondimensionalization, we define the following control parameters for our simulations
\begin{equation}
\begin{split}
    &\mP = \frac{u_{\rm{ff}} L_s}{\kappa_T},\qquad
    \Ro = \frac{|\alpha|\Delta T}{|\beta|\Delta C},\qquad
    \Pran = \frac{\nu}{\kappa_T},\qquad
    \tau = \frac{\kappa_C}{\kappa_T},\qquad
    \tau_0 = \frac{\kappa_{C,0}}{\kappa_T}
\end{split}
\end{equation}
The dimensionless equations of motion are
\label{sec:simulation_details}
\begin{align}
    &\grad\dot\vec{u} = 0 
        \label{eqn:incompressible} \\
    &\partial_t \vec{u} + \vec{u}\dot\grad\vec{u} = -\grad \varpi + (T - \Ro^{-1}C) \hat{z} + \frac{\Pran}{\mP}\grad^2 \vec{u}
        \label{eqn:momentum} \\
    &\partial_t T + \vec{u}\dot\grad T + w \grad_{\rm{ad}}  + \grad\dot[-\kappa_{T,0} \grad \overline{T}] = \frac{1}{\mP}\grad^2 T'.
        \label{eqn:temperature}, \\
    &\partial_t C + \vec{u}\dot\grad C = -\frac{\tau_0}{\mP}\grad^2\bar{C} + \frac{\tau}{\mP}\grad^2 C',
        \label{eqn:composition}
\end{align}



In this work, we study a three-layer model in $z = [0, 3]$.
We set
\begin{equation}
\gradmu = \begin{cases}
    0       & z \leq 1 \\
    \Ro^{-1}       & 1 < z \leq 2 \\
    0       & 2 < z \\
\end{cases},\qquad
\gradrad = \begin{cases}
    \gradad + 1         & z \leq 2 \\
    \gradad - \Ro^{-1}  & z > 2
\end{cases},
\end{equation}
where in our simple boussinesq system, we define
\begin{equation}
    \gradmu \equiv -\Ro^{-1} \frac{\partial \mu}{\partial z},\qquad
    \gradT \equiv -\frac{\partial T}{\partial z},\qquad
    \gradad = 5 [\Ro^{-1} - 2]\,\,\text{(a constant)}.
\end{equation}
and $\gradT = \gradrad$ if conduction carries all of the energy flux.
We fix the flux carried by convection to be
\begin{equation}
    \Fconv = \kappa_{T,0},
\end{equation}
so the total flux through the system is $F_{\mathrm{tot}} = \Fconv(\gradad + 1)$.
We set $\gradT = \gradrad$ in the initial state.


\section{Simulation Details \& Data Availability}
\label{app:simulation_details}




                                                          	

\bibliographystyle{aasjournal}
\bibliography{biblio}
\end{document}
