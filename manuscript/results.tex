\section{Results}
\label{sec:result}


Volume visualizations of simulation dynamics are shown near the initial state (left) and evolved state (right) in Fig.~\ref{fig:dynamics}.
Buoyancy perturbations normalized by the vertical profile of buoyancy standard deviations are shown in the top two panels.
Vertical velocity is shown in the bottom two panels.
In the initial state, convection occurs in the bottom $\sim1/3$ of the simulation domain; the middle $\sim1/3$ of the domain is stabilized by a composition gradient, and the top $\sim1/3$ is stabilized by a thermal gradient.
The convection excites gravity waves in the stable layers.
The \brunt$\,$ frequency is higher by a factor of 10 in the thermal layer than in the semiconvection layer, so the vertical velocity signature of motions there is smaller than in the semiconvection layer.
Describe overhoot.

The most obvious difference between the panels on the left and the right is that the convection zone has grown in size from $\sim 1/3$ of the simulation domain to $\sim 2/3$ of the simulation domain.
Through continuous overshoot, convection entrained stable, low-composition fluid from the upper region into the convection zone.
This process eroded the composition gradient until the Schwarzschild and Ledoux boundaries of the convection zone were identical.
In other words, the \emph{thermal} stability of the upper zone is sufficient to halt expansion of the convection zone via entrainment, but compositional stability is not.
We see negligible convective penetration (mixing of the bouyancy or entropy profile beyond the sign change in $mathcal{Y}$), but this is expected and part of our experimental design (see appendix).

Figure \ref{fig:profiles} displays vertical profiles that have been averaged horizontally and in time.
Profiles on the left show initial conditions, while profiles on the right show the evolved state.
We show the composition (top panels), the Schwarzschild and Ledoux discriminants (middle panels), and the square \brunt$\,$ and convective frequencies.

In the initial state, we see that the composition is uniform in the CZ ($z \leq 1$) and RZ ($z \geq 2$), but varies linearly in $z \in [1, 2]$ and provides stability.
We also see that the sign change in $\yL$ occurs at $z \sim 1$ while that in $\yS$ occurs at $z \sim 2$.
Finally, we see that $f_{\rm{conv}} = 0$ because we initialize the simulation without any convective velocity.
However, the \brunt$\,$ frequency $N^2$ is negative in a boundary layer at the base of the CZ which drives the instability, and $N^2$ is stable above $z = 1$ (and is more stable by a factor of 10 above $z \sim 2$).

The final state (right) is attained after convection entrains and mixes through the initial composition gradient.
We see that the composition profile (top) is constant in the convection zone, and approximates a step function above the CZ at the top of the overshoot zone. (TODO: Add overshoot to this figure).
In this evolved state, the sign changes in the discriminants $\yL$ and $\yS$ coincide (middle panel).
In the bottom panel, we see that the convective frequency is roughly constant, and see that $N^2 \lesssim 0$ in the bulk CZ.
We can compute the ``stiffness'' $\mS = N^2 / f_{\rm{conv}}^2$ of the radiative-convective boundary by comparing the average CZ value of $f_{\rm{conv}}^2 \sim 10^{-2}$ to the RZ value of $N^2 \sim 10^2$, so $\mS \approx 10^4$.
Boundaries with a low stiffness $\mS \lesssim 10$ easily deform in the presence of convective flows, but convective boundaries in stars often have $\mS \gtrsim 10^6$.
The value of $\mS$ achieved in these simulations is therefore in the right regime to tell us about stars, but these simulations still exhibit more mechanical overshoot than we would expect stars to.


Finally, In figure \ref{fig:kippenhahn}, we plot a Kippenhahn-like diagram of the simulation.
The CZ is shown in orange and is the region below the sign change of both $\mY$ and $\mS$.
The semiconvection zone is shown in green and is the region below the sign changes of $\mY$ and $\mS$.
The RZ is shown in purple and is the region above the sign change of both $\mY$ and $\mS$.
Convection overshoot roughly above the $\yL = 0$ line up to the black line, denoted by a hashed region.
The height of the black line traces out the region where the vertical profile of the convective kinetic energy falls below 10\% of its value in the bulk CZ; this line roughly coincides with the extremum of the composition gradient through the simulation evolution.
Importantly, while the orange line that traces out $\yL = 0$ and the green line tracing out $\yS = 0$ start at different heights, 3D convective motions make these lines converge on long timescales.


