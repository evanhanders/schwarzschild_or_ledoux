\section{Results}
\label{sec:result}


Volume visualizations of the composition field in our simulation are shown near the initial state (left) and evolved state (right) in Fig.~\ref{fig:dynamics}.
While our domain spans $z = [0, 3]$, we only plot $z = [0, 2.5]$ to focus on the interesting regions.
We furthermore plot horizontal lines which correspond to the roots of $\yL$ (orange) and $\yS$ (purple).
Initially, the bottom third of the domain is a CZ ($\yL > 0$ and $\yS > 0$), the middle third is a semiconvection zone ($\yL < 0$ but $\yS > 0$), and the top third is an RZ ($\yL \leq \yS$, $\yS < 0$). 
Note that we observe mechanical overshoot at all times: convective velocities are nonzero a small but appreciable distance above the $\yL = 0$ line.
This occurs because the root of the discriminant denotes where the sign change occurs in the buoyant acceleration, not where the convective velocity is zero.

The most obvious difference between the panels on the left and the right is that the CZ has consumed the semiconvection zone and fills the bottom two-thirds of the box.
Convective flows which overshot the convective boundary entrained low-composition material from the stable layer into the convection zone.
Convective motions mixed this fluid, and this process repeated over thousands of convective overturn times until the Ledoux and Schwarzschild boundaries of the convection zone coincided.
At this point, the convection zone ceased its expansion, and the \emph{thermal} stability of the Schwarzschild-stable RZ was sufficient to halt expansion of the convection zone.

Figure \ref{fig:profiles} displays vertical profiles from our simulation in the initial (left) and evolved (right) states.
Shown are the composition $\mu$ (top), the discriminants $\yL$ and $\yS$ (middle), and the square \brunt$\,$ frequency (top) as well as the square convective frequency defined as
\begin{equation}
f_{\rm{conv}}^2 = \frac{|\vec{u}|^2}{L_{\rm{conv}}^2},
\label{eqn:fconv2}
\end{equation}
where $\vec{u}$ is the velocity and $L_{\rm{conv}}$ is the depth of the convectively unstable layer.

Initially, the composition is uniform in the CZ ($z \leq 1$) and RZ ($z \geq 2$), but varies linearly in the semiconvection zone $z \in [1, 2]$.
The root of $\yL$ occurs at $z \approx 1$ while that of $\yS$ occurs at $z \approx 2$.
Finally, initially $f_{\rm{conv}}^2 = 0$ because we start in a stationary state.
The \brunt$\,$ frequency $N^2$ is negative in a boundary layer at the base of the CZ which drives the instability.
$N^2$ is stable for $z \gtrsim 1$, and is larger in the RZ than the semiconvection zone by an order of magnitude\footnote{We ran simulations where $N^2$ was identical in the RZ and semiconvection zone and saw similar behavior.
We make $N^2$ large in the RZ to reduce overshoot in the evolved state.}

The evolved state is attained after convection entrains and mixes the stabilizing fluid in the semiconvection zone.
We see that the composition profile (top) is constant in the convection zone, and approximates a step function at the top of the overshoot zone.
The roots of the discriminants $\yL$ and $\yS$ coincide (middle).
Furthermore, in the CZ, the convective frequency is roughly constant and $N^2 \lesssim 0$.
In the RZ, $f_{\rm{conv}}^2 \approx 0$ and $N^2 \gg 0$.
We can compute the ``stiffness'' of the radiative-convective interface,
\begin{equation}
\mS = \frac{N^2|_{\rm{RZ}}}{f_{\rm{conv}}^2|_{\rm{CZ}}},
\label{eqn:stiffness}
\end{equation}
which is related to the oft-studied Richardson number.
In our evolved simulation, we measure $\mS \sim 10^{4}$.
Boundaries with a low stiffness $\mS \lesssim 10$ easily deform in the presence of convective flows, but convective boundaries in stars often have $\mS \gtrsim 10^6$.
The value of $\mS$ achieved in these simulations is therefore in the right regime to study entrainment at a stellar convective boundary.

Finally, in Figure \ref{fig:kippenhahn}, we plot a Kippenhahn-like diagram of the simulation's height vs.~time to show evolutionary trends.
The roots of $\yL$ and $\yS$ are respectively shown as orange and purple lines.
The CZ is colored orange and sits below the root of $\yL$, the RZ is colored purple and sits above the root of $\yS$, and the semiconvection zone is colored green and sits between the roots.
Convection motions ``overshoot'' above the root of $\yL$.
The height where the horizontally-averaged kinetic energy falls below 10\% of its bulk-CZ value is marked with a black line, and the hashed region below it is the overshoot zone.
We note that the black line and overshoot zone roughly correspond with the maximum of $\partial\mu/\partial z$ (Fig.~\ref{fig:profiles}, upper right), so this is a good description.
Importantly, note that the lines tracing $\yL = 0$ and $\yS = 0$ start at different heights, but 3D convective mixing makes these lines converge on long timescales.


