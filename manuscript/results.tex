\section{Results}
\label{sec:result}

\begin{itemize}
\item We simulate a three-layer system with X properties using the Dedalus pseudospectral solver.
We refer the reader to appendices~\ref{app:model} and \ref{app:simulation_details} for details of the model assumptions, setup, and numerical methods.
\item In figure \ref{fig:dynamics} we show volume visualizations of instantaneous dynamics near the beginning, during the entrainment phase, and in the evolved state of one of these simulations.
Here is a description of the things to look for in these visualizations.
I should write this before I decide which things to put into this figure.
\item In figure \ref{fig:profiles} we show horizontally- and time-averaged profiles of the initial state of the simulation, during the entrainment phase, and in the evolved state.
Here's an in-depth description of what's plotted.
Here's what you should take away from each panel of the plots.
\item In figure \ref{fig:kippenhahn}, we plot a Kippenhahn-like diagram of the simulation.
The orange region is the convection zone, the green region is Ledoux-stable but Schwarzschild-unstable, and the purple region is both Schwarzschild and Ledoux stable.
The line at the top of the orange region determines the boundary of the convection zone according to $y_L$, and the line at the bottom of the purple region determines the CZ boundary according to $y_S$.
While these lines start at different heights, convection entrains low-composition fluid according to a classic $\sqrt{t}$ entrainment law until these criteria are the same.
\end{itemize}

%\begin{figure}[t]
%\centering
%\includegraphics[width=\columnwidth]{grad_profiles.pdf}
%\caption{
%Horizontally- and temporally-averaged profiles of the thermodynamic gradients from the simulation in Fig.~\ref{fig:vertical_dynamics_panels}.
%We plot $\justgrad$ (green) compared to $\gradad$ (purple, a constant) and $\gradrad$ (orange); note the extended penetration zone (PZ) where $\justgrad \approx \gradad > \gradrad$.
%    \editone{
%        The dashed vertical line denotes the Schwarzschild boundary of the convection zone (CZ), the solid vertical line denotes the bottom of the radiative zone (RZ), and the greyed region denotes the PZ-RZ boundary layer.
%    }
%\label{fig:grad_profiles}
%}
%\end{figure}


