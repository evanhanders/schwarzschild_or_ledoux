\section{Results}
\label{sec:results}

In Fig.~\ref{fig:dynamics}, we visualize the composition field in our simulation near the initial state (left) and evolved state (right).
Thick horizontal lines denote the convective boundaries per the Ledoux (orange, $\yL = 0$) and Schwarzschild (purple, $\yS=0$) criteria.
Initially, the bottom third of the domain is a CZ, the middle third is an SZ, and the top third is an RZ.
Convection motions extend beyond $\yL = 0$ at all times; we refer to these motions as overshoot \citep[which is discussed in][]{korre_etal_2019}.
Overshoot occurs because the Ledoux boundary is not the location where convective velocity is zero, but rather the location where buoyant acceleration changes sign due to a sign change in the entropy gradient.

The difference between the left and right panels demonstrates that the CZ consumes the SZ.
Overshooting convective motions entrain low-composition material into the CZ where it is homogenized.
This process increases the size of the CZ and repeats over thousands of convective overturn times until the Ledoux and Schwarzschild criteria predict the same convective boundary.
After this entrainment phase, the convective boundary stops moving.
The boundary is stationary because the radiative flux renews the stable temperature gradient; there is no analogous process to reinforce the composition gradient\footnote{Nuclear timescales are generally much longer than dynamical timescales and can be neglected as a source of composition.}.

In Fig.~\ref{fig:profiles}, we visualize vertical profiles in the initial state (left) and evolved state (right).
Shown are the composition $\mu$ (top), the discriminants $\yL$ and $\yS$ (middle), and two important frequencies (bottom): the square \brunt$\,$ frequency $N^2$ and the square convective frequency,
\begin{equation}
f_{\rm{conv}}^2 = \frac{|\vec{u}|^2}{\ell_{\rm{conv}}^2},
\label{eqn:fconv2}
\end{equation}
with $|\vec{u}|$ the horizontally-averaged velocity magnitude and $\ell_{\rm{conv}}$ the depth of the Ledoux-unstable layer.

The composition is initially uniform in the CZ (${z \lesssim 1}$) and RZ ($z \gtrsim 2$), but varies linearly in the SZ ($z \in [1, 2]$).
We have $\yL(z\approx1) = 0$ but $\yS(z\approx2) = 0$.
An unstable boundary layer at the base of the CZ drives the instability and has negative $N^2$.
For $z \gtrsim 1$, we have positive $N^2$, which is larger in the RZ than the SZ.
We found similar results in simulations where $N^2$ was constant across the RZ and SZ.

In the evolved state (right panels), the composition (top) is well-mixed in the CZ and hashed overshoot zone, but decreases abruptly at the top of the overshoot zone.
We take the height where the horizontally-averaged kinetic energy falls below 10\% of its bulk-CZ value to be the top of the overshoot zone.
In this state, the Schwarzschild and Ledoux criteria agree upon the location of the convective boundary (middle).

The rate at which the CZ entrains the SZ depends on the stiffness of the radiative-convective interface,
\begin{equation}
\mS = \frac{N^2|_{\rm{RZ}}}{f_{\rm{conv}}^2|_{\rm{CZ}}},
\label{eqn:stiffness}
\end{equation}
which is related to the Richardson number ${\rm{Ri} = \sqrt{\mS}}$.
The time to entrain the SZ is roughly ${\tau_{\rm{entrain}} \sim (\delta h / \ell_{\rm{c}})^2 \rm{R}_{\rho}^{-1} \mS \tau_{\rm{dyn}}}$ \citep[per][eqn.~3]{fuentes_cumming_2020}, where $\delta h$ is the depth of the SZ, $\ell_{\rm{c}}$ is the characteristic convective length scale, $\Ro \in [0, 1]$ is the density ratio \citep[see][eqn.~7]{garaud_2018}, and $\tau_{\rm{dyn}}$ is the dynamical timescale which in our simulation is the convective overturn timescale.
In Fig.~\ref{fig:profiles}, bottom right panel, we have $f_{\rm{conv}}^2|_{\rm{CZ}} \approx 3 \times 10^{-3}$ and $N^2|_{\rm{RZ}} \approx 10^2$, so $\mS \approx 3 \times 10^4$.
Convective boundaries in stars often have $\mS \gtrsim 10^6$, so our simulation is in the same high-$\mS$ regime as stars.
The value of $\mathrm{R}_{\rho}$ can vary greatly throughout the depth of an SZ in a star; we use $\rm{R}_\rho = 1/10$.
The relevant evoluationary timescale during the main sequence is the nuclear time $\tau_{\rm{nuc}}$.
Since $\tau_{\rm{nuc}}/\tau_{\rm{dyn}} \gg (\delta h/\ell_{\rm{c}})^2 \mS/\Ro$ even for $\mS \sim 10^6$, SZs should be immediately entrained by bordering CZs on the main sequence and during other evolutionary stages in which convection reaches a steady state.
Note that while values of $\Ro \ll 1$ increase $\tau_{\rm{entrain}}$, they also support efficient mixing by ODDC (see Sec.~\ref{sec:conclusions}).

Finally, in Fig.~\ref{fig:kippenhahn} we display a Kippenhahn-like diagram of the simulation's evolution.
This diagram demonstrates how the CZ, SZ, and RZ boundaries evolve.
The convective boundary measurements are shown as orange ($\yL = 0$) and purple ($\yS = 0$) lines.
The CZ is colored orange and fills the region below the Ledoux boundary, the RZ is colored purple and fills the region above the Schwarzschild boundary, and the SZ is colored green and fills the region between these boundaries.
Convection motions overshoot beyond the Ledoux boundary into a hashed overshoot zone, which we define identically to the one displayed in Fig.~\ref{fig:profiles}.
The top of the overshoot zone (black line) correspond with the edge of the well-mixed region (Fig.~\ref{fig:profiles}, upper right).
While the Schwarzschild and Ledoux boundaries start at different heights, 3D convective mixing causes them to converge.

We briefly note that we performed additional simulations with the same initial stratification as in Fig.~\ref{fig:profiles} (left), but with lower values of $\mS$, higher and lower values of $\Ro$, and less turbulence (lower Reynolds number), and the evolutionary trends described here are present in all simulations.


