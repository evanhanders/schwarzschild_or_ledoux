\documentclass[onecolumn, amsmath, amsfonts, amssymb]{aastex62}
\usepackage{mathtools}
\usepackage{natbib}
\usepackage{bm}
\newcommand{\vdag}{(v)^\dagger}
\newcommand\aastex{AAS\TeX}
\newcommand\latex{La\TeX}


\newcommand{\Div}[1]{\ensuremath{\nabla\cdot\left( #1\right)}}
\newcommand{\DivU}{\ensuremath{\nabla\cdot\bm{u}}}
\newcommand{\angles}[1]{\ensuremath{\left\langle #1 \right\rangle}}
\newcommand{\KS}[1]{\ensuremath{\text{KS}(#1)}}
\newcommand{\KSstat}[1]{\ensuremath{\overline{\text{KS}(#1)}}}
\newcommand{\grad}{\ensuremath{\nabla}}
\newcommand{\RB}{Rayleigh-B\'{e}nard }
\newcommand{\stressT}{\ensuremath{\bm{\bar{\bar{\Pi}}}}}
\newcommand{\lilstressT}{\ensuremath{\bm{\bar{\bar{\sigma}}}}}
\newcommand{\nrho}{\ensuremath{n_{\rho}}}
\newcommand{\approptoinn}[2]{\mathrel{\vcenter{
	\offinterlineskip\halign{\hfil$##$\cr
	#1\propto\cr\noalign{\kern2pt}#1\sim\cr\noalign{\kern-2pt}}}}}

\newcommand{\appropto}{\mathpalette\approptoinn\relax}
\newcommand{\mR}{\ensuremath{\mathcal{R}}}
\newcommand{\mP}{\ensuremath{\mathcal{P}}}
\newcommand{\bu}{\ensuremath{\bm{u}}}
\newcommand{\bV}{\ensuremath{\bm{\omega}}}
\newcommand{\cross}[2]{\ensuremath{#1 \times #2}}
\newcommand{\dotp}[2]{\ensuremath{#1 \cdot #2}}
\newcommand{\curl}[1]{\ensuremath{\cross{\grad}{\left(#1\right)}}}
\newcommand{\pderiv}[2]{\ensuremath{\frac{\partial #1}{\partial #2}}}
\newcommand{\xHe}{\ensuremath{X_{\text{He}}}}

\renewcommand{\bar}[1]{\ensuremath{\overline{#1}}}
\renewcommand{\vec}[1]{\ensuremath{\mathbf{#1}}}
\renewcommand{\dot}{\ensuremath{\cdot}}

\usepackage{color}
\newcommand{\gv}[1]{{\color{blue} #1}}

\begin{document}
\section{The question: Schwarzschild or Ledoux?}
The fundamental question behind this project is not, ``If you look at an instantaneous profile of a star, which stability criterion should you use?''
We know the answer to that question: Ledoux.
The question here is, ``is the Ledoux criterion \emph{fragile}?''
Or, given time, will a Ledoux-stable (but Schwarzschild-unstable) region which is adjacent to a convection zone remain stable over dynamical timescales?

\section{The quasi-Boussinesq equations of motion}
\subsection{Dimensional equations}
We will solve the Boussinesq equations of motion in the incompressible limit.
We will solve for the velocity (\vec{u}), the temperature ($T$), and the composition ($\mu$).
In this limit, density variations are ignored except in the buoyant term in the momentum equation, where they follow
\begin{equation}
\frac{\rho}{\rho_0} = \alpha T + \beta \mu,\qquad\qquad
\alpha \equiv \frac{\partial\ln\rho}{\partial T} \qquad\mathrm{and}\qquad
\beta  \equiv \frac{\partial\ln\rho}{\partial \mu}
\end{equation}
We will use values of $\beta > 0$ and $\alpha < 0$ (higher concentration $\rightarrow$ denser, lower temperature $\rightarrow$ denser).
The Boussinesq momentum equation (in which $\rho$ can only vary on the gravitational term) is,
\begin{equation}
\partial_t \vec{u} + \vec{u}\dot\grad\vec{u} + \frac{\grad p}{\rho_0} = \frac{\rho}{\rho_0}\vec{g} + \nu\grad^2\bm{u}.
\end{equation}
Removing $\rho$ under the Boussinesq approximation, we retrieve our evolution equations,
\begin{align}
\partial_t \vec{u} + \vec{u}\dot\grad\vec{u} + \frac{\grad p}{\rho_0} &= (\alpha T + \beta \mu)\vec{g} + \nu \grad^2\vec{u} \\
    \partial_t T + \vec{u}\dot(\grad T - \grad T_{\mathrm{ad}}) &= \grad\dot[k_0 \grad \bar{T}] + \chi \grad^2 T' + Q \label{eqn:temperature_eqn} \\
    \partial_t \mu + \vec{u}\dot\grad \mu &= \chi_{\mu,0} \grad^2 \bar{\mu} + \chi_{\mu} \grad^2 \mu' .
\end{align}
Here, $k_0$ and $\chi_{\mu,0}$ are diffusivities which act on the horizontally averaged mode while $\chi$ and $\chi_\mu$ act on the fluctuations; $\nu$ is a viscosity which acts on all flows.
Assuming that a vertical energy flux $F$ is carried through the system, this allows us to define an adiabatic and radiative gradient,
\begin{equation}
    \grad \equiv - \grad T, \qquad
    \grad_{\rm{ad}} \equiv - \grad T_{\rm{ad}}, \qquad
    \grad_{\rm{rad}} \equiv \frac{F}{k_0}.
\end{equation}
Note that in this system, if you linearize and idealize and solve for the dispersion relation, the squared brunt frequencies (assuming $\vec{g} = - g \hat{z}$) are
\begin{equation}
    N_{\mathrm{therm}}^2 = \alpha g (\grad - \grad_{\rm{ad}}), \qquad
    N_{\mathrm{comp}}^2 =  -\beta  g \grad \mu, \qquad
    N^2 = N_{\mathrm{therm}}^2 + N_{\mathrm{comp}}^2,
\end{equation}
and the stability criterion is to have $N^2 > 0$.

%\section{Stellar Parameters}
%
%\subsection{Basics}
%We now nondimensionalize. 
%We take the length scale, $L$, to be the depth of the RZ, the helium gradient region.
%We take the velocity scale, $u_{\mathrm{ff}}$, to be the freefall velocity of the convection, which in turn sets the timescale.
%We take the magnitude of the temperature, $\Delta T$, to be the magnitude of the \emph{convective temperature pertubations}, which are small compared to the background.
%We take the magnitude of the composition, $\Delta Y$, to be the difference between $Y$ at the top and bottom of the RZ.
%Two of these quantities can be immediately read off of the MESA model, and they are $L = 4.955 \times 10^{8}$ cm and $\Delta Y = 0.3369$.
%Note also that the background density will be taken to be the density at the CZ-RZ interface of $\rho_0 = 9.4 \times 10^3$ g/cm$^3$.
%
%We will next turn our attention to $\Delta T$, $u_{\mathrm{ff}}$, and $\tau$.
%By our choice of nondimensionalization, 
%\begin{equation}
%-\frac{\Delta T}{L} = \grad T_0 - \grad_{\mathrm{ad}} = \frac{N_{\mathrm{therm}^2}}{\alpha g}.
%\end{equation}
%From the ideal gas equation of state, $\alpha = - T^{-1}$, and we will take reference values of $T = 9.02 \times 10^7$ K and $g = 6.4 \times 10^6$ cm/s$^2$ at the CZ-RZ interface.
%Furthermore, within the portion of the CZ that we are modeling, $N_{\mathrm{therm}}^2 = -7.43 \times 10^{-12}$ s$^{-2}$.
%Combining these values, we find 
%\begin{align}
%\Delta T &= \frac{L N_{\mathrm{therm}^2}}{\alpha g} = 5.17 \times 10^{-2}\,\mathrm{K},\\
%u_{\mathrm{ff}} &= \sqrt{\alpha g L \Delta T} = 1.35 \times 10^{3}\,\mathrm{cm/s},\\
%\tau &= L/u_{\mathrm{ff}} = 3.67 \times 10^{5} \mathrm{s}.
%\end{align}
%
%
%Next, we need to understand how stable the RZ is compared to the CZ.
%This is commonly quoted through an ``effective density ratio,''
%\begin{equation}
%\mathrm{Ro} = \frac{\alpha \Delta T}{\beta \Delta Y} = \frac{-N_{\mathrm{therm}}^2}{N_{\mathrm{comp}}^2 } = 5.82 \times 10^{-9} \sim \mathrm{Ma}^2,
%\end{equation}
%where I have used $N_{\mathrm{comp}}^2 = 0.128$ as the average value of $N^2$ in the RZ of the MESA model.
%In general, $N_{\mathrm{comp}}^2$ can be arbitrarily large, because the composition gradient can be as large as it needs to be.
%In this problem $N_{\mathrm{comp}}^2$ in the RZ is of order $(c_s/L)^2$ because the composition gradient is generated by tuning $\grad_{\mathrm{rad}}$ to equal $\grad_{\mathrm{ad}}$ (which forces $\partial_r(\mu) \sim -1/L$). 
%So, in this problem, it will be useful to note that there is a direct relationship between Ro and the Mach number of the convection,
%\begin{equation}
%\mathrm{Ro} \approx \mathrm{Ma}^2,
%\end{equation}
%and we will use $\mathrm{Ma}^2$ as our variable going forward.
%
%\subsection{Diffusivities}
%
%Next, we look at diffusivities.
%I went into more detail in this in an old set of notes, but here I'm just going to use the results.
%The thermal diffusivity can be found from the radiative conductivity,
%\begin{equation}
%\chi = \frac{16 \sigma_{SB} T^3}{3 \rho_0^2 c_p \kappa} = 6.55 \times 10^{4} \,\mathrm{cm}^2/\mathrm{s},
%\end{equation}
%where $c_p = 1.17 \times 10^{8}$ erg g$^{-1}$ K${-1}$ at the CZ-RZ interface and the opacity is $\kappa = 0.329$ cm$^2$/g there.
%The viscous diffusivity and molecular diffusivity are a bit more complicated, but referring to my previous document which derived those formulas, and making the caveat that these formulae are for pure-helium fluids, the diffusivities are roughly,
%\begin{align}
%\nu &= (2.2 \times 10^{-15}\mathrm{cm}^2/\mathrm{s})\frac{T^{5/2}}{\rho}\left\{23 - \ln 8 - \frac{1}{2}\ln(2 n) +\frac{3}{2}\ln T\right\}^{-1} = 1.106 \,\mathrm{cm}^2/\mathrm{s} \\
%\chi_Y &= (3.75 \times 10^{-16}\mathrm{cm}^2/\mathrm{s})\frac{T^{5/2}}{\rho} = 3.083 \mathrm{cm}^2/\mathrm{s}
%\end{align}
%where $T$ and $\rho$ are in cgs units and $n = \rho/(\mu m_p) = 3.36 \times 10^{27}$ cm$^{-3}$ at the CZ-RZ interface.
%From these, we can make various estimates of the turbulence in the flows,
%\begin{equation}
%\mathrm{Re}_{\mathrm{ff}} = \frac{u_{\mathrm{ff}} L}{\nu} = 6.05 \times 10^{11},\qquad
%\mathrm{Pe}_{\mathrm{ff}} = \frac{u_{\mathrm{ff}} L}{\chi} = 1.02 \times 10^7,\qquad
%\mathrm{Re}_Y             = \frac{u_{\mathrm{ff}} L}{\chi_Y} = 2.17 \times 10^{11}.
%\end{equation}
%Clearly it's impossible to simulate anything close to those, so it's probably safe to just set $\mathrm{Re}_{\mathrm{ff}} = \mathrm{Pe}_{\mathrm{ff}} = \mathrm{Re}_Y$ and make them as large as possible (or at least modestly large) in our simulations.
%I don't expect the amount of turbulence in the problem to cause big differences, and so this should be fine.
%However, I do expect the magnitude of $\mathrm{Pe}_{\mathrm{ff}}$ to perhaps be important, because this value tells us how quickly heat can diffuse through the stable layer (and thus how effectively energy generated by the excess fuel can escape!).

\subsection{Nondimensional Equations}
The nondimensional equations are then (with $\varpi$ the dynamical pressure),
\begin{align}
    \partial_t \vec{u} + \vec{u}\dot\grad\vec{u} + \grad \varpi  &= (-T + \rm{R}_d Y)\hat{z} + \frac{1}{\mathrm{Re}}\grad^2 \vec{u},\\
    \partial_t T - \bm{u}\dot(\grad - \grad_{\mathrm{ad}}) &= - \grad\cdot[k_0 \bar{\grad}] + \frac{1}{\rm{Pe}} \grad^2 T' + Q,\\
    \partial_t \mu + \bm{u}\dot\grad \mu &= \frac{1}{\mathrm{Re}_{\mu,0}}\grad^2 \bar{\mu} + \frac{1}{\mathrm{Re}_{\mu}}\grad^2 \mu'.
\end{align}
The nondimensional brunt frequencies are  
\begin{equation}
    N_{\mathrm{therm}}^2 = (\grad_{\rm{ad}} - \grad), \qquad
    N_{\mathrm{comp}}^2 =  -\rm{R}_d\grad \mu, \equiv \grad_\mu \qquad
    N^2 = N_{\mathrm{therm}}^2 + N_{\mathrm{comp}}^2 = (\grad_\mu + \grad_{\rm{ad}} - \grad).
\end{equation}


\section{Three-layer experiment}
We want to set up an experiment where there are three layers, characterized by:
\begin{enumerate}
    \item (CZ) $\grad = \grad_{\rm{ad}}$, $\grad_\mu = 0$, $\grad_{\rm{rad}} > \grad_{\rm{ad}}$.
    \item (semiconvection) $\grad = \grad_{\rm{rad}}$, $\grad_{\rm{rad}} - \grad_\mu < \grad_{\rm{ad}}$, $\grad_{\rm{rad}} > \grad_{\rm{ad}}$.
    \item (RZ) $\grad = \grad_{\rm{rad}}$, $\grad_\mu = 0$, $\grad_{\rm{rad}} < \grad_{\rm{ad}}$.
\end{enumerate}

As in our previous work, we will set the convective flux $F_{\rm{conv}} = Q \delta_H = 0.2$ with $Q = 1$ and $\delta_H = 0.2$.
We will set the flux entering the bottom of the domain, and thus we will set $k_0$, so that $F_{\rm{bot}} = k_0 \grad_{\rm{ad}} = \eta F_{\rm{conv}}$.
By our choice of $Q$ and by setting the convective domain size to $L = 1$, we implicitly set $f_{\rm{conv}} \sim 1$.
To set the stiffness $\mathcal{S}$, we set the value of $N^2 = \grad_{\rm{ad}} - \grad_{\rm{rad}} = \mathcal{S}$ in the RZ.
We have still not chosen a value of $\grad_{\rm{ad}}$ (just its relation to $\grad_{\rm{rad}}$ in the RZ and $\grad_{\rm{rad}}$, or $k_0$, in the CZ).
So we will choose a value of the penetration parameter $\mathcal{P}$ that is either realistic of stellar values (1) or which has no penetration ($\ll 1$).
This parameter is
\begin{equation}
    \mathcal{P} = -\frac{k_{\rm{CZ}} (\grad_{\rm{rad}} - \grad_{\rm{ad}})_{\rm{CZ}}}
    {k_{\rm{RZ}} (\grad_{\rm{rad}} - \grad_{\rm{ad}})_{\rm{RZ}}}.
\end{equation}
We furthermore want to set it so that $\Delta \mu = 1$ (that's the magnitude of the $\mu$ decrease over zone 2 above).
We additionally want to ensure $N^2 > 0$ in zone 2, so $\rm{R}_d$ must be set so that $\grad_\mu > \grad_{\rm{rad}} - \grad_{\rm{ad}}$ in the CZ.
Under these constraints, we are free to choose whatever values of Re, Re$_\mu$, and Pe we want.
We want to choose them, probably, so that the semiconvective layer is \emph{not unstable} to any classical semiconvective instabilities.
We should prove that it is stable to these instabilities using a 1-layer model which just encompasses zone (2) above.
Once we know it's stable, then we should do the 3-layer model.


\end{document}
