\documentclass[onecolumn, amsmath, amsfonts, amssymb]{aastex62}
\usepackage{mathtools}
\usepackage{natbib}
\usepackage{bm}
\newcommand{\vdag}{(v)^\dagger}
\newcommand\aastex{AAS\TeX}
\newcommand\latex{La\TeX}


\newcommand{\Div}[1]{\ensuremath{\nabla\cdot\left( #1\right)}}
\newcommand{\DivU}{\ensuremath{\nabla\cdot\bm{u}}}
\newcommand{\angles}[1]{\ensuremath{\left\langle #1 \right\rangle}}
\newcommand{\KS}[1]{\ensuremath{\text{KS}(#1)}}
\newcommand{\KSstat}[1]{\ensuremath{\overline{\text{KS}(#1)}}}
\newcommand{\grad}{\ensuremath{\nabla}}
\newcommand{\RB}{Rayleigh-B\'{e}nard }
\newcommand{\stressT}{\ensuremath{\bm{\bar{\bar{\Pi}}}}}
\newcommand{\lilstressT}{\ensuremath{\bm{\bar{\bar{\sigma}}}}}
\newcommand{\nrho}{\ensuremath{n_{\rho}}}
\newcommand{\approptoinn}[2]{\mathrel{\vcenter{
	\offinterlineskip\halign{\hfil$##$\cr
	#1\propto\cr\noalign{\kern2pt}#1\sim\cr\noalign{\kern-2pt}}}}}

\newcommand{\appropto}{\mathpalette\approptoinn\relax}
\newcommand{\mR}{\ensuremath{\mathcal{R}}}
\newcommand{\mP}{\ensuremath{\mathcal{P}}}
\newcommand{\bu}{\ensuremath{\bm{u}}}
\newcommand{\bV}{\ensuremath{\bm{\omega}}}
\newcommand{\cross}[2]{\ensuremath{#1 \times #2}}
\newcommand{\dotp}[2]{\ensuremath{#1 \cdot #2}}
\newcommand{\curl}[1]{\ensuremath{\cross{\grad}{\left(#1\right)}}}
\newcommand{\pderiv}[2]{\ensuremath{\frac{\partial #1}{\partial #2}}}
\newcommand{\xHe}{\ensuremath{X_{\text{He}}}}

\renewcommand{\vec}[1]{\ensuremath{\mathbf{#1}}}
\renewcommand{\dot}{\ensuremath{\cdot}}

\usepackage{color}
\newcommand{\gv}[1]{{\color{blue} #1}}

\begin{document}
\section{The question: Schwarzschild or Ledoux?}
The fundamental question behind this project is not, ``If you look at an instantaneous profile of a star, which stability criterion should you use?''
We know the answer to that question: Ledoux.
The question here is, ``is the Ledoux criterion \emph{fragile}?''
Or, given time, will a Ledoux-stable (but Schwarzschild-unstable) region which is adjacent to a convection zone remain stable over dynamical timescales?

\section{The dimensional Boussinesq equations of motion}
We will solve the Boussinesq equations of motion in the incompressible limit.
We will solve for the velocity (\vec{u}), the temperature ($T$), and the composition ($\mu$).
In this limit, density variations are ignored except in the buoyant term in the momentum equation, where they follow
\begin{equation}
\frac{\rho}{\rho_0} = \alpha T + \beta \mu,\qquad\qquad
\alpha \equiv \frac{\partial\ln\rho}{\partial T} \qquad\mathrm{and}\qquad
\beta  \equiv \frac{\partial\ln\rho}{\partial \mu}
\end{equation}
We will use values of $\beta > 0$ and $\alpha < 0$ (higher concentration $\rightarrow$ denser, lower temperature $\rightarrow$ denser).
The Boussinesq momentum equation (in which $\rho$ can only vary on the gravitational term) is,
\begin{equation}
\partial_t \vec{u} + \vec{u}\dot\grad\vec{u} + \frac{\grad p}{\rho_0} = \frac{\rho}{\rho_0}\vec{g} + \nu\grad^2\bm{u}.
\end{equation}
Removing $\rho$ under the Boussinesq approximation, we retrieve our evolution equations,
\begin{align}
\partial_t \vec{u} + \vec{u}\dot\grad\vec{u} + \frac{\grad p}{\rho_0} &= (\alpha T - \beta Y)\vec{g} + \nu \grad^2\vec{u} \\
\partial_t T + \vec{u}\dot(\grad T - \grad T_{\mathrm{ad}}) &= \frac{1}{\rho_0 c_v}\grad\dot[(1 + Y)\kappa\grad T] + \frac{\epsilon(Y, T)}{c_v} \label{eqn:temperature_eqn} \\
\partial_t Y + \vec{u}\dot\grad Y &= \chi_Y \grad^2 Y - \frac{1}{\rho_0}\frac{\Delta m_{\mathrm{He}}}{\Delta E}\epsilon(Y, T).
\end{align}
Here, $\kappa = \rho_0 c_p \chi$ is the radiative conductivity where $\chi$ is the radiative diffusivity, and $\grad T_{\mathrm{ad}}$ is the gradient of the adiabatic temperature profile (note that $c_p$ and $c_v$ might be a bit meaningless in the Boussinesq limit, but I'm leaving these dimensional parameters in for retrieving estimates from the MESA model).
Plugging $\kappa$ in and asuming constant $\chi$, the diffusion term in the energy equation becomes 
\begin{equation}
\frac{1}{\rho_0 c_v}\grad\dot[(1 + Y)\kappa\grad T] = \chi\frac{c_p}{c_v} \grad\dot[(1 + Y)\grad T],
\end{equation}
where the $(1 + Y)$ term signifies that radiative conductivity increases with helium concentration.
Finally, $\epsilon(Y, T)$ is the helium burning function, with cgs units [erg cm$^{-3}$ s$^{-1}$], and $\Delta E / \Delta m_{\mathrm{He}}$ is the energy released per mass of helium burned.

Note that in this system, if you linearize and idealize and solve for the dispersion relation, the squared brunt frequencies (assuming $\vec{g} = - g \hat{z}$) are
\begin{equation}
N_{\mathrm{therm}}^2 = -\alpha g (\grad T_0 - \grad_{\mathrm{ad}}), \qquad
N_{\mathrm{comp}}^2 =  \beta  g \grad Y_0, \qquad
N^2 = N_{\mathrm{therm}}^2 + N_{\mathrm{comp}}^2,
\end{equation}
and the stability criterion is to have $N^2 > 0$.

\section{Triple-alpha burning}
The primary nuclear reaction that's occuring is the $3\alpha$ process. 
Per Eqn.~5-104 in Ch.~5.5 of Clayton 1983, the energy generation rate of the triple alpha reaction is
\begin{equation}
\epsilon_{3\alpha} = (3.9 \times 10^{11})\frac{\rho^2 Y^3}{T_8^3} f e^{-42.94/T_8}\,\mathrm{erg}\,\mathrm{g}^{-1}\mathrm{sec}^{-1},
\end{equation}
where $T_8$ is the temperature in units of $10^8$ K and the 3$\alpha$ weak-electron-screening formula is roughly
\begin{equation}
f \approx e^{(2.76 \times 10^{-3})\rho^{1/2}T_8^{-3/2}}.
\end{equation}
Altogether, near $T_8 = 1$, which is the relevant regime for our star,
\begin{equation}
\epsilon_{3\alpha} \approx (4.4 \times 10^{-8}) \rho^2 Y^3 f T_8^{40}\,\mathrm{erg}\,\mathrm{g}^{-1}\mathrm{sec}^{-1}.
\end{equation}
This means that nuclear burning depends very strongly on temperature and somewhat strongly on the Helium mass fraction.
In a perfect world, in Eqn.~\ref{eqn:temperature_eqn}, we would set $\epsilon(Y,T) = \epsilon_{3\alpha}$ directly, but the power on $T$ is a bit too strong for that, so we'll make approximations instead.

Finally, as will be important in the mass fraction equation, the energy liberated per reaction is $\Delta E = 7.274\,\mathrm{Mev} = 1.165 \times 10^{-5}$ erg.
The mass of Helium consumed is that of three alpha particles $\Delta m_{\text{He}} = 3\dot4.0015\,\mathrm{amu}  = 1.993 \times 10^{-23} g$.


\section{Nondimensionalization}

\subsection{Basics}
We now nondimensionalize. 
We take the length scale, $L$, to be the depth of the RZ, the helium gradient region.
We take the velocity scale, $u_{\mathrm{ff}}$, to be the freefall velocity of the convection, which in turn sets the timescale.
We take the magnitude of the temperature, $\Delta T$, to be the magnitude of the \emph{convective temperature pertubations}, which are small compared to the background.
We take the magnitude of the composition, $\Delta Y$, to be the difference between $Y$ at the top and bottom of the RZ.
Two of these quantities can be immediately read off of the MESA model, and they are $L = 4.955 \times 10^{8}$ cm and $\Delta Y = 0.3369$.
Note also that the background density will be taken to be the density at the CZ-RZ interface of $\rho_0 = 9.4 \times 10^3$ g/cm$^3$.

We will next turn our attention to $\Delta T$, $u_{\mathrm{ff}}$, and $\tau$.
By our choice of nondimensionalization, 
\begin{equation}
-\frac{\Delta T}{L} = \grad T_0 - \grad_{\mathrm{ad}} = \frac{N_{\mathrm{therm}^2}}{\alpha g}.
\end{equation}
From the ideal gas equation of state, $\alpha = - T^{-1}$, and we will take reference values of $T = 9.02 \times 10^7$ K and $g = 6.4 \times 10^6$ cm/s$^2$ at the CZ-RZ interface.
Furthermore, within the portion of the CZ that we are modeling, $N_{\mathrm{therm}}^2 = -7.43 \times 10^{-12}$ s$^{-2}$.
Combining these values, we find 
\begin{align}
\Delta T &= \frac{L N_{\mathrm{therm}^2}}{\alpha g} = 5.17 \times 10^{-2}\,\mathrm{K},\\
u_{\mathrm{ff}} &= \sqrt{\alpha g L \Delta T} = 1.35 \times 10^{3}\,\mathrm{cm/s},\\
\tau &= L/u_{\mathrm{ff}} = 3.67 \times 10^{5} \mathrm{s}.
\end{align}


Next, we need to understand how stable the RZ is compared to the CZ.
This is commonly quoted through an ``effective density ratio,''
\begin{equation}
\mathrm{Ro} = \frac{\alpha \Delta T}{\beta \Delta Y} = \frac{-N_{\mathrm{therm}}^2}{N_{\mathrm{comp}}^2 } = 5.82 \times 10^{-9} \sim \mathrm{Ma}^2,
\end{equation}
where I have used $N_{\mathrm{comp}}^2 = 0.128$ as the average value of $N^2$ in the RZ of the MESA model.
In general, $N_{\mathrm{comp}}^2$ can be arbitrarily large, because the composition gradient can be as large as it needs to be.
In this problem $N_{\mathrm{comp}}^2$ in the RZ is of order $(c_s/L)^2$ because the composition gradient is generated by tuning $\grad_{\mathrm{rad}}$ to equal $\grad_{\mathrm{ad}}$ (which forces $\partial_r(\mu) \sim -1/L$). 
So, in this problem, it will be useful to note that there is a direct relationship between Ro and the Mach number of the convection,
\begin{equation}
\mathrm{Ro} \approx \mathrm{Ma}^2,
\end{equation}
and we will use $\mathrm{Ma}^2$ as our variable going forward.

\subsection{Diffusivities}

Next, we look at diffusivities.
I went into more detail in this in an old set of notes, but here I'm just going to use the results.
The thermal diffusivity can be found from the radiative conductivity,
\begin{equation}
\chi = \frac{16 \sigma_{SB} T^3}{3 \rho_0^2 c_p \kappa} = 6.55 \times 10^{4} \,\mathrm{cm}^2/\mathrm{s},
\end{equation}
where $c_p = 1.17 \times 10^{8}$ erg g$^{-1}$ K${-1}$ at the CZ-RZ interface and the opacity is $\kappa = 0.329$ cm$^2$/g there.
The viscous diffusivity and molecular diffusivity are a bit more complicated, but referring to my previous document which derived those formulas, and making the caveat that these formulae are for pure-helium fluids, the diffusivities are roughly,
\begin{align}
\nu &= (2.2 \times 10^{-15}\mathrm{cm}^2/\mathrm{s})\frac{T^{5/2}}{\rho}\left\{23 - \ln 8 - \frac{1}{2}\ln(2 n) +\frac{3}{2}\ln T\right\}^{-1} = 1.106 \,\mathrm{cm}^2/\mathrm{s} \\
\chi_Y &= (3.75 \times 10^{-16}\mathrm{cm}^2/\mathrm{s})\frac{T^{5/2}}{\rho} = 3.083 \mathrm{cm}^2/\mathrm{s}
\end{align}
where $T$ and $\rho$ are in cgs units and $n = \rho/(\mu m_p) = 3.36 \times 10^{27}$ cm$^{-3}$ at the CZ-RZ interface.
From these, we can make various estimates of the turbulence in the flows,
\begin{equation}
\mathrm{Re}_{\mathrm{ff}} = \frac{u_{\mathrm{ff}} L}{\nu} = 6.05 \times 10^{11},\qquad
\mathrm{Pe}_{\mathrm{ff}} = \frac{u_{\mathrm{ff}} L}{\chi} = 1.02 \times 10^7,\qquad
\mathrm{Re}_Y             = \frac{u_{\mathrm{ff}} L}{\chi_Y} = 2.17 \times 10^{11}.
\end{equation}
Clearly it's impossible to simulate anything close to those, so it's probably safe to just set $\mathrm{Re}_{\mathrm{ff}} = \mathrm{Pe}_{\mathrm{ff}} = \mathrm{Re}_Y$ and make them as large as possible (or at least modestly large) in our simulations.
I don't expect the amount of turbulence in the problem to cause big differences, and so this should be fine.
However, I do expect the magnitude of $\mathrm{Pe}_{\mathrm{ff}}$ to perhaps be important, because this value tells us how quickly heat can diffuse through the stable layer (and thus how effectively energy generated by the excess fuel can escape!).

\subsection{Burning}

Finally, we turn our attention to burning.
We will determine the value of burning, as a function of $Y$, both at the top of the CZ (where $T_{\mathrm{top}} = 9.02 \times 10^7$ K) and at the bottom of the CZ (where $T_{\mathrm{bot}} = 1.116 \times 10^8$ K).
We find that
\begin{equation}
\bar{\epsilon}(Y, T) = \frac{\epsilon(Y, T)}{c_v} = (4.4 \times 10^{-8})c_v^{-1}\rho_0^2 f(\rho_0, T_8) T_8^{40} Y^3 \,\mathrm{K}\,\mathrm{s}^{-1}
\end{equation}
and
\begin{align}
\bar{\epsilon}(Y, T_{\mathrm{top}}) = (1.124 \times 10^{-9}) Y^3\,\,\,\,\mathrm{K}\,\mathrm{s}^{-1}, \\
\bar{\epsilon}(Y, T_{\mathrm{bot}}) = (5.095 \times 10^{-6}) Y^3\,\,\,\,\mathrm{K}\,\mathrm{s}^{-1}.
\end{align}
Numerically, we will use a simplified burning function.
We can represent this burning function in the dimensionless temperature equation as
\begin{equation}
\frac{\tau}{\Delta T}\bar{\epsilon} = Q Y T^{n}
\end{equation}
where $n$ is some positive integer (say $n = 5$) and where $Q$ sets the magnitude of the burning at the bottom of the CZ,
\begin{equation}
Q = \frac{\tau}{\Delta T} \Delta Y (5.095 \times 10^{-6} \mathrm{   K / s}) \approx 12.
\end{equation}
And, in the mass fraction equation, we have
\begin{equation}
\tau c_v \frac{\Delta m_{\mathrm{He}}}{\Delta E} \bar{\epsilon} = -C_r Y T^n,
\end{equation}
where $C_r$ is the consumption rate, whose value at the bottom of the CZ is roughly
\begin{equation}
C_r \approx 8 \times 10^{-11},
\end{equation}
meaning that it takes very many convective overturn times for the fuel to be consumed.
This may as well just be $C_r = 0$, and so for simplicitly we will assume in our models that Helium is never consumed.

\subsection{Nondimensional Equations}
The nondimensional equations are then (with $\varpi$ the dynamical pressure),
\begin{align}
\partial_t \vec{u} + \vec{u}\dot\grad\vec{u} + \grad \varpi  &= (T + \mathrm{Ma}^{-2}Y)\hat{z} + \frac{1}{\mathrm{Re}_{\mathrm{ff}}}\grad^2 \vec{u},\\
\partial_t T + \bm{u}\dot(\grad T - \grad_{\mathrm{ad}}) &= \frac{1}{\mathrm{Pe}_{\mathrm{ff}}}\grad\dot[(1 + Y)\grad T] + Q \xi(Y,T),\\
\partial_t Y + \bm{u}\dot\grad Y &= \frac{1}{\mathrm{Re}_Y}\grad^2 Y - C_r \xi(Y,T).
\end{align}
In a star, the burning function $\xi(Y,T) \propto Y^3 T^{40}$, but we will use a simpler function in our simulations.
Stellar values of parameters are:
\begin{enumerate}
\item $\mathrm{Ma}^2 \approx 6 \times 10^{-9}$,
\item $\mathrm{Re}_\mathrm{ff} \approx 6 \times 10^{11}$,
\item $\mathrm{Pe}_\mathrm{ff} \approx 10^7$,
\item $\mathrm{Re}_Y \approx 2 \times 10^{11}$,
\item $Q   \approx 12$
\item $C_r \approx 8 \times 10^{-11}$.
\end{enumerate}


\section{Numerical Model}
We will set up a numerical model where the O(1) temperature fluctuations of the convection are a factor of Ma$^2$ smaller than the background adiabatic gradient.
This background state will also also satisfy thermal equilibrium.
For the Helium mass fraction, we will assume it is zero in the convection zone and it has a constant gradient in the radiative zone.
An initial state which satisfies these conditions is
\begin{equation}
Y_0 = 
\begin{cases}
0,	 	&	z < 0 \\
z, 		& 	0 \leq z \leq 1 \\
\end{cases},\qquad
\grad T_0 = -\frac{\mathrm{Ma}^{-2}}{1 + Y_0}, \qquad
\grad_{\text{ad}} = 
\begin{cases}
1 - \mathrm{Ma}^{-2},				& 	\bar{Y} = 0 \\
\grad T,		&	\bar{Y} \geq 0.02 \\
\end{cases}.
\label{eqn:model}
\end{equation}
In general, our spectral methods dislike really sharp, piecewise functions like these, so we will take advantage of error functions to build smooth models.
We define
\begin{equation}
E_{1-0}(z, z_0, \delta) = \frac{1}{2}\left[1 - \text{erf}\left(\frac{z - z_0}{\delta}  \right)    \right]
\qquad\text{and}\qquad
E_{0-1}(z, z_0, \delta) = 1 - E_{1-0}(z, z_0, \delta),
\end{equation}
where $E_{1-0}$ is a function that varies smoothly from 1 to 0, and $E_{0-1}$ varies smoothly from 0 to 1.
In all of our models, we set $\delta = 0.05$.

We thus construct the $Y$ profile through its derivative by specifying:
\begin{equation}
\frac{ \partial Y_0}{\partial z} = E_{0-1}(z, 0, \delta),
\end{equation}
and we integrate this under the constraint that $Y_0(z = -1) = 0$ to get the $Y_0$ profile.
$\partial_z T_0$ is constructed with this integrated $Y_0$ profile, and then integrated to find $T_0$ under the constraint that $T_0 = 0$ at the top of the model.
An example of the initial $Y_0$ profile in a dedalus simulation is shown in Fig.~\ref{fig:initial_Y}.
%\begin{figure}
%\centering
%\includegraphics[width=0.75\textwidth]{Y_frac_star_and_dedalus.png}
%\caption{(Top panel) The Helium mass fraction in the star shortly before breathing pulses ($Y_{CZ} \approx 0.14$).
%		 (Bottom panel) The Helium mass fraction in the initial conditions of the Dedalus simulation.
%	\label{fig:initial_Y} }
%\end{figure}


Stability is described by the square brunt-vaisaila frequency,
$$
N^2 = \grad T - \grad_{\text{ad}} + \mathrm{Ma}^{-2}\grad Y.
$$
The stability of the initial conditions are shown in Fig.~\ref{fig:brunt}.
%\begin{figure}
%\centering
%\includegraphics[width=0.75\textwidth]{brunt_star_and_dedalus.png}
%\caption{(Top panel) The brunt frequency in the star.
%		 (Bottom panel) The brunt frequency in the Dedalus initial conditions.
%		 In both panels, a dashed line represents a negative value.
%		 The weird behavior of the Dedalus thermal brunt shows where the value of $Y_0 \geq 0.02$ (see Eqn.~\ref{eqn:model}).
%	\label{fig:brunt} }
%\end{figure}


The burning function is defined as
\begin{equation}
\xi(Y, T) = 
\begin{cases}
0									& T < T_0(z=0)\\
g(Y, T)								& T \geq T_0(z=0) \,\,\&\,\, g(Y, T) \leq 1 \\
1									& g(Y, T) > 1
\end{cases},
\qquad
g(Y, T) = Y \left(2\frac{T - T_{0}(z=0)}{T_0(z=-1) - T_0(z=0)}\right)^n.
\label{eqn:burning}
\end{equation}
Here, $n$ is some power that doesn't make the simulations impossible to resolve like $n = 2$.
The ``2'' in the definition of $g$ makes it so that $\xi$ maxes out at a value of 1 at $z = -0.5$.
In Fig.~\ref{fig:burning}, the value of $\xi$ is shown, with $n = 2$, for the initial conditions, if the star were to have $Y = 1$ everywhere.
%\begin{figure}
%\centering
%\includegraphics[width=0.75\textwidth]{burning_func_dedalus.png}
%\caption{The value of $\xi$, defined in Eqn.~\ref{eqn:burning}, for the initial conditions, if the star were to have $Y = 1$ everywhere.
%	\label{fig:burning} }
%\end{figure}








\end{document}
