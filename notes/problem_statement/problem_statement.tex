\documentclass[onecolumn, amsmath, amsfonts, amssymb]{aastex62}
\usepackage{mathtools}
\usepackage{natbib}
\usepackage{bm}
\newcommand{\vdag}{(v)^\dagger}
\newcommand\aastex{AAS\TeX}
\newcommand\latex{La\TeX}


\newcommand{\Div}[1]{\ensuremath{\nabla\cdot\left( #1\right)}}
\newcommand{\DivU}{\ensuremath{\nabla\cdot\bm{u}}}
\newcommand{\angles}[1]{\ensuremath{\left\langle #1 \right\rangle}}
\newcommand{\KS}[1]{\ensuremath{\text{KS}(#1)}}
\newcommand{\KSstat}[1]{\ensuremath{\overline{\text{KS}(#1)}}}
\newcommand{\grad}{\ensuremath{\nabla}}
\newcommand{\RB}{Rayleigh-B\'{e}nard }
\newcommand{\stressT}{\ensuremath{\bm{\bar{\bar{\Pi}}}}}
\newcommand{\lilstressT}{\ensuremath{\bm{\bar{\bar{\sigma}}}}}
\newcommand{\nrho}{\ensuremath{n_{\rho}}}
\newcommand{\approptoinn}[2]{\mathrel{\vcenter{
	\offinterlineskip\halign{\hfil$##$\cr
	#1\propto\cr\noalign{\kern2pt}#1\sim\cr\noalign{\kern-2pt}}}}}

\newcommand{\appropto}{\mathpalette\approptoinn\relax}
\newcommand{\mR}{\ensuremath{\mathcal{R}}}
\newcommand{\mP}{\ensuremath{\mathcal{P}}}
\newcommand{\bu}{\ensuremath{\bm{u}}}
\newcommand{\bV}{\ensuremath{\bm{\omega}}}
\newcommand{\cross}[2]{\ensuremath{#1 \times #2}}
\newcommand{\dotp}[2]{\ensuremath{#1 \cdot #2}}
\newcommand{\curl}[1]{\ensuremath{\cross{\grad}{\left(#1\right)}}}
\newcommand{\pderiv}[2]{\ensuremath{\frac{\partial #1}{\partial #2}}}
\newcommand{\xHe}{\ensuremath{X_{\text{He}}}}

\renewcommand{\bar}[1]{\ensuremath{\overline{#1}}}
\renewcommand{\vec}[1]{\ensuremath{\mathbf{#1}}}
\renewcommand{\dot}{\ensuremath{\cdot}}

\usepackage{color}
\newcommand{\gv}[1]{{\color{blue} #1}}

\begin{document}
\section{The question: Schwarzschild or Ledoux?}
The fundamental question behind this project is not, ``If you look at an instantaneous profile of a star, which stability criterion should you use?''
We know the answer to that question: Ledoux.
The question here is, ``is the Ledoux criterion \emph{fragile}?''
Or, given time, will a Ledoux-stable (but Schwarzschild-unstable) region which is adjacent to a convection zone remain stable over dynamical timescales?

\section{The Boussinesq equations of motion}
\subsection{Dimensional equations}
We will solve the Boussinesq equations of motion in the incompressible limit.
We will solve for the velocity (\vec{u}), the temperature ($T$), and the composition ($\mu$).
In this limit, density variations are ignored except in the buoyant term in the momentum equation, where they follow
\begin{equation}
\frac{\rho}{\rho_0} = -\alpha T + \beta \mu,\qquad\qquad
\alpha \equiv \bigg|\frac{\partial\ln\rho}{\partial T}\bigg| \qquad\mathrm{and}\qquad
\beta  \equiv \bigg|\frac{\partial\ln\rho}{\partial \mu}\bigg|
\end{equation}
By our sign conventions and definitions, $\beta$ and $\alpha$ are positive and density is increased by high concentrations or by low temperatures.
The Boussinesq equations of motion are
\begin{equation}
\end{equation}
Removing $\rho$ under the Boussinesq approximation, we retrieve our evolution equations in cartesian coordinates,
\begin{align}
    \partial_t \vec{u} + \vec{u}\dot\grad\vec{u} + \frac{\grad p}{\rho_0} &= -\frac{\rho}{\rho_0}g\vec{e}_z + \nu_0\grad^2\bar{\vec{u}}  + \nu \grad^2\vec{u}\\
    \partial_t T + \vec{u}\dot(\grad T - \grad T_{\mathrm{ad}}) &= \grad\dot[\kappa_{T,0} \grad \bar{T}] + \kappa_T \grad^2 T' \label{eqn:temperature_eqn} \\
    \partial_t \mu + \vec{u}\dot\grad \mu &= \kappa_{\mu,0} \grad^2 \bar{\mu} + \kappa_{\mu} \grad^2 \mu' .
\end{align}
Here, $\nu_0$, $\kappa_{T,0}$, and $\kappa_{\mu,0}$ are diffusivities which act on the horizontally averaged mode while $\nu$, $\kappa_T$ and $\kappa_\mu$ act on the fluctuations.
Assuming that a vertical energy flux $F$ is carried through the system, this allows us to define an adiabatic and radiative gradient,
\begin{equation}
    \grad \equiv - \grad T, \qquad
    \grad_{\rm{ad}} \equiv - \grad T_{\rm{ad}}, \qquad
    \grad_{\rm{rad}} \equiv \frac{F}{\kappa_{T,0}}.
\end{equation}
Note that in this system, if you linearize and idealize and solve for the dispersion relation, the squared brunt frequencies (assuming $\vec{g} = - g \hat{z}$) are
\begin{equation}
    N_{\mathrm{therm}}^2 = \alpha g (\grad - \grad_{\rm{ad}}), \qquad
    N_{\mathrm{comp}}^2 =  -\beta  g \grad \mu, \qquad
    N^2 = N_{\mathrm{therm}}^2 + N_{\mathrm{comp}}^2,
\end{equation}
and the stability criterion is to have $N^2 > 0$.


\subsection{Nondimensionalization}
We will nondimensionalize on the length scale of the convective domain, the superadiabaticity of the radiative gradient in the convective domain, the composition gradient, and the timescale of freefall over the convective domain.
We define the following nondimensional parameters:
\begin{align}
    \rm{Pr} = \frac{\nu}{\kappa_T},\qquad
    \tau    = \frac{\kappa_{\mu}}{\kappa_T},\qquad
    \rm{Pr}_0 = \frac{\nu_0}{\kappa_{T,0}},\qquad
    \tau_0    = \frac{\kappa_{\mu,0}}{\kappa_{T,0}},\\
    \rm{Pe} = \frac{u_{\rm{ff}} L}{\kappa_T},\qquad
    \rm{R}_0^{-1} = \frac{N_{\rm{comp}}^2}{|N_{\rm{structure}}^2|} = \frac{\beta \Delta \mu}{\alpha \Delta T},\\
    u_{\rm{ff}} = L/t = \sqrt{\alpha g L \Delta T},\qquad
    \Delta T = L|\partial_z T - \partial_z T_{\rm{ad}}|,\qquad
    \Delta \mu = L|\partial_z \mu|,\qquad
    f = \kappa_{T,0}/\kappa_T
\end{align}


The nondimensional equations are then (with $\varpi$ the dynamical pressure),
\begin{align}
    \partial_t \vec{u} + \vec{u}\dot\grad\vec{u} + \grad \varpi &= (T - \rm{R}_0^{-1})\vec{e}_z + \frac{1}{\text{Pe}} \left(f\,\text{Pr}_0 \grad^2\bar{\vec{u}} + \rm{Pr}\grad^2\vec{u}'\right)\\
    \partial_t T + \vec{u}\dot(\grad T - \grad T_{\mathrm{ad}}) &= \frac{1}{\rm{Pe}}\left(\grad\dot[f \grad \bar{T}] + \grad^2 T'\right)\label{eqn:temperature_eqn} \\
    \partial_t \mu + \vec{u}\dot\grad \mu &= \frac{1}{\rm{Pe}}\left(f \, \tau_0 \grad^2 \bar{\mu} + \tau \grad^2 \mu' \right).
\end{align}
The nondimensional brunt frequencies are  
\begin{equation}
    N_{\mathrm{therm}}^2 = (\grad_{\rm{ad}} - \grad), \qquad
    N_{\mathrm{comp}}^2 =  -\rm{R}_d^{-1}\grad \mu, \equiv \grad_\mu \qquad
    N^2 = N_{\mathrm{therm}}^2 + N_{\mathrm{comp}}^2 = (\grad_\mu + \grad_{\rm{ad}} - \grad).
\end{equation}


\section{Three-layer experiment}
We want to set up an experiment where there are three layers, characterized by:
\begin{enumerate}
    \item (CZ) $\grad = \grad_{\rm{ad}}$, $\grad_\mu = 0$, $\grad_{\rm{rad}} > \grad_{\rm{ad}}$.
    \item (semiconvection) $\grad = \grad_{\rm{rad}}$, $\grad_{\rm{rad}} - \grad_\mu < \grad_{\rm{ad}}$, $\grad_{\rm{rad}} > \grad_{\rm{ad}}$.
    \item (RZ) $\grad = \grad_{\rm{rad}}$, $\grad_\mu = 0$, $\grad_{\rm{rad}} < \grad_{\rm{ad}}$.
\end{enumerate}

Per the work that Pascale has done, there are three regimes of stability for the region in 2:
\begin{enumerate}
    \item (Ledoux unstable) $\rm{R}_0^{-1} < 1$.
    \item (oscillatory doubly-diffusive convetion [ODDC]) $1 < \rm{R}_0^{-1} < (\rm{Pr} + 1)/(\rm{Pr} + \tau)$.
    \item (Ledoux stable) $\rm{R}_0^{-1} > (\rm{Pr} + 1)/(\rm{Pr} + \tau)$.
\end{enumerate}
We want to set up an experiment where the semiconvection layer is in regime 3, that is, where it is truly Ledoux unstable.

The composition gradient is
\begin{equation}
    \frac{\partial \mu}{\partial z} = 
    \begin{cases}
    0   & z \in [0, 1] \\
    -1 & z \in [1, 2] \\
    0 & z > 2
    \end{cases},
\end{equation}
and we set $\mu = 0$ at the upper boundary (so it's 1 in the lower CZ).


\end{document}
